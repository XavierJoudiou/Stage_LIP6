\section{Prise en compte de la mobilité: Blue Banana}
	\label{BlueBanana}
	Il nous a été possible de voir, dans les chapitres précédents, des mécanismes qui permettent de faire évoluer le système en réaction à des événements. Solipsis ne pourrait pas bien fonctionner dans un monde avec des traces correspondant à la réalité, Blue Banana va donc mettre en place un mécanisme d'anticipation des mouvements des avatars pour mieux s'adapter.
	\par Blue Banana présente une solution aux différents problèmes que peut rencontrer une architecture pair à pair dans les MMOGs. La mobilité des avatars implique de nombreux échanges de données à travers le réseau pair à pair. Comme les overlays de l'état de l'art n'anticipent pas cette mobilité, les données nécessaires ne seront pas chargées à temps, ce qui conduit à des défaillances transitoires au niveau applicatif. Blue Banana a été réalisée pour résoudre ce problème, il modélise et prédit les mouvements des avatars ce qui permet à l'overlay de s'adapter par anticipation aux besoins du jeux.
	\subsection{Solutions introduites}
	Blue Banana est implémenté au dessus de Solipsis qu'il a été possible d'étudier dans le chapitre~\ref{solipsis}. Plusieurs observations ont été faites et différentes optimisations en sont ressorties. Il nous a été possible d'observer plusieurs types de zone (dense ou non, cf.~\ref{trace}) et que les mécanismes d'adaptation sont trop tardifs pour être mis en place dans la réalité (le chargement des données sera trop lent).
	\subsubsection{Les états de l'avatar}
	\label{Automate}
	Une des premières innovations qui a été introduite est la distinction de plusieurs états d'un avatar. Comme il a été possible de voir dans le chapitre sur la collecte de traces, un avatar se comporte différemment en fonction des zones du monde. Trois états ont donc été introduits:
	\begin{itemize}
	\renewcommand{\labelitemi}{$\bullet$}
		\item \textbf{H}(alted): l'avatar est immobile.
		\item \textbf{T}(ravelling): l'avatar se déplace rapidement sur la carte et il a une trajectoire droite.  
		\item \textbf{E}(xploring): l'avatar est en train d'explorer une zone, sa trajectoire est confuse et sa vitesse est lente.
	\end{itemize} 
	Le changement d'état de l'avatar se fait en fonction de la vitesse de celui-ci, si la vitesse devient supérieure à une borne définie et que l'avatar est dans l'état E alors l'avatar passe en état T. Ce modèle pourrait être affiné par la suite en prenant en compte l'accélération ou l'historique des mouvements. Sur la figure~\ref{automateMob}, nous pouvons mieux distinguer les différents changements d'état. Chaque nœud va agir en fonction de cette automate, il sera initialisé à l'état \textbf{H}(alted). \\
	

	\begin{figure}[!h]
        \centering
        \includegraphics[scale=0.4]{./Ressources/Images/automate.png}
        \caption{\textit{\small Automate décrivant les mouvements d'un avatar. \textbf{En gras}: le nom de la transition, en \textit{italique} sa sémantique}}
        \label{automateMob}
        \end{figure}




	\subsubsection{Anticipation des mouvements}
	Un autre mécanisme a été mis en place, il s'agit d'anticiper les mouvements d'un avatar, pour cela deux suppositions sont faites: seulement une prédiction courte est cohérente, et plus l'avatar se déplace rapidement, plus il y a de chance qu'il continue dans la même direction~\cite{191}. Comme nous pouvons voir sur la figure~\ref{Propa_Algo}, en fonction du vecteur de mouvement de l'avatar, le nœud B, s'il est dans l'état \textbf{T}, va chercher des nœuds qui se trouvent sur la trajectoire probable de l'avatar, tant que son ensemble de voisins n'est pas plein. Le nœud B va envoyer un message aux voisins qui sont le plus près de lui par rapport au vecteur de mouvement. Un mécanisme pour évaluer si le nœud n'est pas trop près, et donc rapatrier des données ne servirait pas car le temps des communications serait supérieur au temps du déplacement de l'avatar. Un des risques est de rapatrier des nœuds qui seront inutiles si l'avatar va changer de direction ou d'état. L'amélioration de ce point fait parti des futures pistes pour améliorer l'algorithme d'anticipation des mouvements.\\
	\vspace{5mm}
        \begin{figure}[!h]
        \centering
        \includegraphics[scale=0.5]{./Ressources/Images/propagation_algo.png}\\
        \caption{Algorithme de propagation}
        \label{Propa_Algo}
        \end{figure}
        \vspace{5mm}
	\subsection{Expérimentations et Résultats}
		Le travail a été testé sur le simulateur à évènements discrets PeerSim ~\cite{peersim}, les expérimentations ont eu pour objectif de comparer Solipsis avec et sans Blue Banana.
		\subsubsection{Le simulateur PeerSim et la description des expérimentations}
		\par PeerSim est un simulateur de réseau pair à pair, qui a deux modes de fonctionnement: par cycles ou par évènements. C'est une API riche et modulaire qui est codée en Java, c'est une composante du projet BISON de l'université de Bologne (Italie). Ce simulateur permet de simuler un large nombre de machines et de tester différentes configurations du réseau. Le simulateur va faire des simplifications sur les couches réseau et les contraintes physiques (latence, pannes, ...). Chaque nœud est considéré comme un module qui va échanger des messages avec les autres nœuds du système. La plupart des plateformes de simulation est basée sur le modèle à évènements discrets. Il est possible de distinguer deux entités: les nœuds et les messages. Le temps va seulement évoluer à chaque nouvel évènement sur un nœud.
		%\subsubsection{Description des expérimentations}
		\par Au départ de la simulation, une carte initiale des traces est introduite dans le simulateur, la carte provient d'une étude de La et Michiardi~\cite{LM-wosn08} dans Second Life. Ensuite, le simulateur va initialiser l'overlay de Solipsis et vérifier que les deux règles de Solipsis sont bien respectées sur chaque nœud, nous insérerons ensuite le reste des traces. Il faut aussi régler les différents paramètres du simulateur (nombre d'avatar, surface du monde, densité, accélération des avatars, vitesse de connection, etc). Plusieurs métriques sont mises en place pour évaluer les résultats:
	\begin{itemize}
	\renewcommand{\labelitemi}{$\bullet$}
		\item \textit{Violation of Solipsis fundamental rules}: Regarde si les propriétés de \textit{Global Connectivity} et de \textit{Local Awareness} sont respectées.
		\item \textit{Knowledge of nodes ahead of the movement}: Mesure pour les avatars qui se déplacent rapidement, le temps moyen pour qu'il connaisse un nœud qui sera sur sa trajectoire.
		\item \textit{Exchanged messages count}: Mesure l'impact de Blue Banana sur le réseau, cela va compter le nombre de messages introduits par Blue Banana et Solipsis.
	\end{itemize}
		\subsubsection{Les résultats}
		Les résultats les plus intéressants montrent que Blue Banana diminue les transitions en échec de 55\% à 20\%, augmentent la connaissance des prochains nœuds de 270\% et cela en créant un overhead de seulement 2\%. Les résultats montrent que le mécanisme d'anticipation introduit par Blue Banana aide l'overlay de Solipsis à s'adapter à temps et à réduire significativement le nombre de violation des règles de Solipsis (de 55\% ou 80\% à 20\%). 
