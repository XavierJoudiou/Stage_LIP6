\section{Introduction}
	A travers ce rapport, nous allons commencer par observer les différents travaux déjà réalisés sur les applications pair à pair, nous nous intéresserons dans le détail aux jeux vidéos massivement multijoueur. Ces applications impliquent que différentes propriétés soient vérifiées si l'on veut qu'elles soient utilisées par un grand nombre de personnes en même temps et sur une longue durée. L'architecture pair à pair peut répondre efficacement à certaines des différentes propriétés nécessaires au bon fonctionnement des applications mais elle pose un problème au niveau de la latence. Le but du stage a été d'améliorer les liens du réseau pair à pair pour que l'utilisateur puisse avoir dans son voisinage les données qui lui seront nécessaires aux itérations suivantes, pour cela il faut anticiper au mieux les mouvements du joueur.\\

	Tout d'abord nous expliquerons pourquoi l'approche pair à pair est celle qui paraît la plus adaptée pour répondre aux différentes problématiques qu'induisent ces applications, nous en profiterons pour rappeler rapidement les caractéristiques des différentes architectures (cf \ref{whyp2p}, page \pageref{whyp2p}). Les différentes études des traces des avatars dans les environnements virtuels seront ensuite expliquées (cf \ref{trace}, page \pageref{trace}). Cette étude des traces, nous permettra de mieux expliquer les différentes solutions d'amélioration. Ensuite, les mécanismes permettant une meilleure mise en place de l'architecture pair à pair seront expliqués, nous rentrerons un peu plus dans les détails pour Solipsis car Blue Banana l'utilise comme base (cf \ref{solipsis}, page \pageref{solipsis}). Après avoir parlé des différents mécanismes déjà existants, nous parlerons alors du travail Blue Banana qui a permis de mettre en place une première amélioration (cf \ref{BlueBanana}, page \pageref{BlueBanana}). \\
	
	Ensuite, nous présenterons les deux principales améliorations que nous avons mis en place durant ce stage. Nous commencerons par présenter la solution que consiste à intégrer un cache dans les nœuds du réseau, pour ensuite expliquer une amélioration du prefetch qui a été mis en place dans Blue Banana. Nous observerons les résultats de chacune des solutions et expliquerons les différentes pistes explorées qui se sont révélées infructueuses. Nous récapitulerons les évolutions que nous avions envisagées et qui pourrait permettre de continuer le travail actuel.

