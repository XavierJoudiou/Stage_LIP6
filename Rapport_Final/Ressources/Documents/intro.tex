\section{Introduction}
	Le début de ce rapport va permettre d'observer les différents travaux déjà réalisés sur les applications pair à pair. Les jeux vidéos massivement multijoueur étant le principal type d'application qui va nous intéresser. Ces applications, regroupant un grand nombre de personnes, impliquent que différentes propriétés (jouabilité, fluidité, réactivité, etc) soient vérifiées. L'architecture pair à pair peut répondre efficacement à certaines des différentes propriétés nécessaires au bon fonctionnement des applications mais un problème peut se poser au niveau de la latence. Le but du stage a été d'améliorer le voisinage logique du réseau pair à pair pour que l'utilisateur puisse avoir dans son voisinage les données qui lui seront nécessaires aux instants suivantes dans l'environnement, pour cela il faut anticiper au mieux les mouvements du joueur.\\

	Tout d'abord nous expliquerons pourquoi l'approche pair à pair est celle qui paraît la plus adaptée pour répondre aux différentes problématiques qu'induisent ces applications, nous en profiterons pour rappeler rapidement les caractéristiques des différentes architectures (cf \ref{whyp2p}, page \pageref{whyp2p}). Ensuite, les mécanismes permettant une meilleure mise en place de l'architecture pair à pair seront expliqués, nous rentrerons un peu plus dans les détails pour Solipsis (cf \ref{solipsis}, page \pageref{solipsis}). Solipsis est un travail qui propose un monde virtuel entièrement décentralisé et scalable. Un Overlay qui est caractérisé par une forte malléabilité applicative, sert de support au monde virtuel. Après avoir parlé des différents mécanisme existants qui ne prennent pas en compte la mobilité, une étude des traces des avatars dans les environnements virtuels sera alors introduite (cf \ref{trace}, page \pageref{trace}). Cette étude des traces permettra de mieux expliquer les différentes solutions d'amélioration de la réactivité du réseau pair à pair. Ensuite, le travail Blue Banana qui a permis de mettre en place une première amélioration, au fonctionnement du monde virtuel proposé par Solipsis, sera expliqué (cf \ref{BlueBanana}, page \pageref{BlueBanana}). Blue Banana permet d'anticiper les mouvements des avatars, pour ainsi adapter au mieux le voisinage des nœuds. \\
	
	Ensuite, les deux principales améliorations, qui ont été mises en place durant ce stage, seront présentées. La première solution consiste à intégrer un cache dans les nœuds du réseau, son fonctionnement et ses performances seront expliqués. L'autre solution, implémentés durant le stage, est une amélioration du préchargement des données qui a été mis en place dans Blue Banana. Outre l'explication des résultats de chacune des solutions, les différentes implémentations explorées seront présentées. Enfin les autres pistes d'amélioration, qui ont été envisagées, seront expliquées.

