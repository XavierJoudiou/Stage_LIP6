\section{Conclusion}
	Durant le stage, nous avons étudié les différentes architectures disponibles pour les MMOGs, nous avons observé les limites et les avantages que peuvent avoir les architectures client/serveur et pair à pair. Ensuite grâce à l'étude des traces, nous avons pu dégager différents points permettant l'amélioration de la prise en compte de la mobilité dans les MMOGs. Nous avons vu différents mécanismes permettant d'améliorer la réactivité dans les applications pair à pair, et plus précisément dans les MMOGs, mais la plupart fonctionnait, au mieux, en réaction aux évènements~\cite{10.1109/SRDS.2006.33}. Blue Banana est un mécanisme d'anticipation des mouvements, ce qui permet de mieux faire évoluer le réseau. Nous avons développé des solutions d'amélioration, en nous appuyant sur ce qui a été fait dans Blue Banana.\\

	\par Les deux solutions mises en place nous ont permis de travailler sur deux problèmes distincts. La mise en place du cache a permis d'essayer de trouver une amélioration à un endroit qui n'avait pas encore été modifié. Les modifications apportées au préchargement des données ont améliorées la solution mise en place dans Blue Banana. 	
