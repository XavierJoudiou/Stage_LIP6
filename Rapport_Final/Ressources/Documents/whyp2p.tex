\section{Vers des solutions distribuées pour les MMOGs}
	\label{whyp2p}
	Il est devenu important, pour les éditeurs de MMOGs, d'étudier les différentes architectures en amont de la réalisation du jeu pour ainsi gérer au mieux les différentes dépenses ~\cite{14101410} (maintenance des serveurs, ajout de add-ons,etc). Une étude des différences entre les architectures Client/Serveur et pair à pair va permettre de mettre en évidence les différences des deux solutions, et les avantages et inconvénients d'une approche pair à pair. Les différentes raisons qui nous feraient passer d'une architecture Client/Serveur à une architecture pair à pair, seront alors étudiées. 

	\subsection{Architectures actuelles pour les MMOGs}
	\par Dans la plupart des jeux en ligne massivement multijoueur, l'architecture est de type client/serveur (voir page~\pageref{P2P/ClServ}). Dans cette architecture, il y a une forte distinction entre le client, qui envoie des requêtes au serveur et attend les réponses, et le serveur qui est à l'écoute de requêtes des clients. Cette approche simplifie la sécurité et le fonctionnent global des jeux. Par exemple, pour effectuer des mises à jour sur l'état global du jeu, il suffit de le faire sur une seule machine (le serveur principal qui sera potentiellement répliqué) et il y a peu de problème d'incohérence entre les données. De même pour l'administration et le contrôle des tricheurs, toutes les données étant regroupées sur une seule machine, le contrôle sera plus simple que dans des systèmes décentralisés. \\
	\par Un des problèmes que peut rencontrer l'architecture client/serveur est qu'elle aura des difficultés pour passer à l'échelle sans déployer un grand nombre de machines puissantes et coûteuses. Il y un risque d'apparition d'un goulot d'étranglement et si un trop grand nombre de joueurs se connecte, le serveur pourrait avoir des difficultés à accepter toute la charge des traitements à effectuer~\cite{1198269}. Ce problème est résolu en ayant des serveurs de très grandes capacités ou en mettant en place des clusters de serveurs. Ces solutions induisent un gros investissement dès le début de la mise en service du jeu et le coût de maintenance est élevé. Il est donc difficile de mettre cela en place pour des applications \textit{open source} ou ayant peu de moyens, c'est surtout pour celles-ci que les architectures pair à pair peuvent être intéressantes dans l'immédiat. Un autre problème est la disponibilité du système en cas de panne du serveur. Si le serveur tombe en panne alors plus personne n'aura accès à l'application que ce dernier faisait fonctionner, à la différence d'une architecture répartie. \\
	Au vu du nombre croissant de participants à ce genre de jeux vidéos massivement multijoueurs, le passage à l'échelle devient un sujet très important et c'est pour cela que les recherches sur des architectures distribuées sont de plus en plus importantes (voir la figure~\ref{stat_P2P}). \\
	\vspace{5mm} 
        \begin{figure}[!h]
        \centering
        \includegraphics[scale=0.85]{./Ressources/Images/Stat_Rech_P2P.png}\\
        \caption{Tableau montrant l'évolution des recherches dans le domaine du P2P (MMOGs et Overlay)}
        \label{stat_P2P}
        \end{figure}
	\subsection{Le pair à pair, la solution?}
	\par L'augmentation croissante des recherches sur le sujet atteste du fait que les architectures actuelles comportent des limites, et que l'architecture pair à pair peut apparaitre comme une alternative intéressante. Le problème du passage à l'échelle est un des facteurs les plus importants, et donc une des raisons principales de toutes ces recherches. Les architectures pair à pair ne font plus ressortir d'entité serveur et client comme nous le connaissions dans l'architecture client/serveur. Chaque nœud sera client et serveur en fonction du temps et en fonction de ses besoins (voir la figure~\ref{P2P/ClServ}). Les systèmes pair à pair peuvent avoir une multitude d'utilisations, que ce soit dans le partage de fichier~\cite{gnutella,napster,kazaa}, la communication~\cite{skype}, les jeux vidéos~\cite{starwars}, le calcul scientifique~\cite{Pastry,xtremweb,chord}, le militaire~\cite{jxta}, etc. \\
	\par L'architecture pair à pair est faite telle qu'il n'y ait pas de goulot d'étranglement, car nous passons d'un système où tout transitait par un point unique à un système qui comporte un grand nombre d'entités qui peuvent toutes avoir le même rôle. Le suppression du goulot d'étranglement va modifier l'architecture, les données devront être envoyées à plusieurs entités au lieu d'une seule auparavant. L'utilisation de l'architecture pair à pair va alors entrainer un plus grand nombre de communications que l'architecture Client/Serveur. De plus, l'architecture pair à pair nécessite des communicatons entre les entités, de la gestion des ressources partagées et d'autres problèmes liés à la répartition des données. Il faut donc trouver des solutions à tous ces problèmes éventuels. Le pair à pair peut donc être plus adapté, dans certains cas, à des applications massivement multijoueurs mais il faudra conserver les propriétés nécessaires au fonctionnement des applications. \\
		%\subsubsection{Inconvénients}
	\par Les systèmes pair à pair sont plus difficiles à surveiller, les phénomènes de tricherie sont donc plus difficiles à détecter, de même pour la sécurité. Une étude des phénomènes de tricherie~\cite{1198269} explique qu'il existe trois types de tricherie: par Confidentialité, c'est à dire obtenir des informations non autorisées sur d'autres utilisateurs; par Intégrité, si il y des modifications du monde, des lois physiques ou les lois du jeu non autorisées; par Disponibilité, c'est le fait de provoquer des ralentissements ou des arrêts de partie du jeu (référence vers Challenges in P2P gaming). Il faut que le système soit aussi fiable sur le long terme et qu'il soit tolérant aux connexions et déconnexions (churn).\\
	\par Les jeux vidéos sont des applications distribuées qui ne sont pas dites ``critiques'' (temps réel ``mou''), le fait que le jeu ralentisse légèrement de manière très ponctuelle n'est pas très gênant. Certaines propriétés des applications distribuées peuvent être retardées ou sautées ponctuellement. Les jeux vidéos ont des avantages qui font qu'il sera moins ardu de passer d'une architecture Client/Serveur à une architecture pair à pair~\cite{1267692}:
	\begin{itemize}
		\renewcommand{\labelitemi}{$\bullet$}
		\item Les jeux vidéos tolèrent une cohérence relaché, pour les différents états de l'application, en comparaison à des applications militaires ou civiles dans des secteurs à risques (risques humains, financiers, matériels, etc).
		\item La prédiction des écritures et des lectures peut se faire avec une certaine limite, grâce à l'ensemble des règles définies dans le jeu. Le but du stage sera l'amélioration de la prise en compte de la mobilité dans les MMOGs, dont Blue Banana~\cite{191} sera un point de départ possible.
	\end{itemize}
	\vspace{1cm}
	\begin{figure}[!h]
	\centering
	\includegraphics[scale=0.45]{./Ressources/Images/p2p-85145.png}\\
	\caption{Schéma des architectures pair à pair et client/serveur}
	\label{P2P/ClServ}
	\end{figure} 
\newpage
