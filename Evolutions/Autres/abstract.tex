\noindent{\textbf{Résumé}}\\
	\par \textit{Depuis plusieurs années, un nouveau type d'architecture des systèmes est apparu. Il s'agit de l'architecture pair à pair, cette architecture est devenue populaire grâce à des applications de partage de fichiers. Nous allons nous intéresser aux jeux vidéos massivement multijoueur (MMOG pour Massively Multiplayer Online Games) qui sont de plus en plus populaires et qui font ressortir des problèmes que l'architecture pair à pair doit pouvoir corriger. Le problème du passage à l'échelle sera l'un des plus importants à résoudre pour permettre à un grand nombre de joueurs de participer simultanément. Nous verrons comment l'architecture pair à pair peut être une des solutions.\\ 
	 Pour remédier à cela, une solution consiste à remplacer le modèle client/serveur par un réseau logique pair à pair (overlay). Malheureusement, les protocoles pair à pair existants sont trop peu réactifs pour assurer la faible latence nécessaire à ce genre d’applications. Néanmoins, quelques travaux ont déjà été menés pour résoudre ce problème. L’idée est d’adapter le voisinage de chaque pair afin que toute l’information dont il aura besoin dans l'avenir se trouve proche de lui dans le réseau. Il est alors nécessaire de correctement évaluer les futurs besoins de chaque pair, et de faire évoluer son voisinage à temps. Dans ce document, nous allons présenter différentes idées pour tenter d'améliorer la réactivité des réseaux pair à pair dans les MMOG.}\\
