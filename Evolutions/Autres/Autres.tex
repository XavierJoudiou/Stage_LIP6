\documentclass[11pt,a4paper]{article}
%\documentclass[10pt,twocolumn]{article}

\usepackage{Autres} %% cibler doc/modules/
\setlength{\columnsep}{1cm}


%\onecolumn
\begin{document}
  \fairetitre{Amélioration de la réactivité des réseaux pair à pair pour les MMOGs}{Les différentes idées d'amélioration}{Xavier Joudiou}{Sergey Legtchenko \& Sébastien Monnet}{17/06/10}

\newpage
%\onecolumn
\tableofcontents
\newpage
%\twocolumn

\input{abstract}

\section{Introduction}
Ce document va récapituler les différentes solutions d'amélioration de la réactivité des réseaux pair à pair pour les MMOG. Il faudra tenir compte de la difficultés et du temps de mise en place de chaque solution, pour choisir la solution qui sera réalisable avant la fin du stage et qui puisse apporter des améliorations non négligeables.
Plusieurs pistes sont envisagées dont celle menant aux mouvements de groupe, que l'on peut aussi retrouver de façon plus précises dans un rapport bibliographique qui lui est dédié.

\section{Mise en place d'un cache pour les zones peuplés}

\section{Mécanismes de connaissance des routes entre les Hotspots}

\section{Amélioration du prefetch de Blue Banana}
Une des solutions est d'essayer d'améliorer le travail déjà réalisé dans Blue Banana~\cite{191}, l'objectif serait prefetcher plus finement les nœuds. 
\section{Les mouvements de groupe}

\section{Autres}



\newpage
\bibliographystyle{plain}
\bibliography{Biblio}


 

\end{document}
