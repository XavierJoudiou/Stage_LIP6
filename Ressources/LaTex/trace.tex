\section{Collecte des traces des utilisateurs}
	Expliquer les différentes techniques de collecte de trace.
	\subsection{Les différentes techniques de récupération de trace}
	Dans la littérature, il y a plusieurs études des traces des utilisateurs dans des différents environnements virtuels. La plupart vont récupérer les traces des avatars sur des jeux vidéos tel que World Of Warcraft et Second Life ( \textbf{Ajouter Biblio} ). Ces travaux vont permettre de bien comprendre les différents comportements des joueurs, ces travaux nous permettront de détecter les différents comportement en fonction des zones plus ou moins peuplées. Différentes mesures sont apparaître comme: le nombre de joueurs, le nombre d'arrivée et de départ, le temps moyen d'une session, la distribution des joueurs, etc. \\
	Certaines techniques se servent d'un boot qui est introduit dans le jeu et qui va récupérer des informations sur les autres joueurs. Le boot va récupérer des informations à intervalle régulier, nous pourrons ainsi analyser les déplacements et vérifier les modèles.  \\
	Limitations de la limite de la collecte des traces !! \\
	

	\subsection{Observations des traces}
	La collecte des traces ...
