\section{Collecte des traces des utilisateurs}
	\label{trace}
	Expliquer les différentes techniques de collecte de trace.
	\subsection{Les différentes techniques de récupération de trace}
	Dans la littérature, il y a plusieurs études des traces des utilisateurs dans des différents environnements virtuels. La plupart vont récupérer les traces des avatars sur des jeux vidéos tel que World Of Warcraft et Second Life ( \textbf{Ajouter Biblio} ). Ces travaux vont permettre de bien comprendre les différents comportements des joueurs, ces travaux nous permettront de détecter les différents comportement en fonction des zones plus ou moins peuplées. Différentes mesures vont apparaître comme: le nombre de joueurs, le nombre d'arrivée et de départ, le temps moyen d'une session, la distribution des joueurs, etc. L'étude des traces des joueurs permet aussi de détecter les tricheurs qui utilisent des boots. \\
	
	Certaines techniques se servent d'un boot qui est introduit dans le jeu et qui va récupérer des informations sur les autres joueurs. Le boot va récupérer des informations à intervalle régulier, nous pourrons ainsi analyser les déplacements et vérifier les modèles. Par exemple dans {Avatar Mobility in Networked Virtual Environments: Measurements, Analysis, and Implications}, une librairie open source \textit{libsecondlife} a été développée pour mettre en place le boot dans Second Life. Pour vérifier que les données récupérées sont cohérentes, une technique de positionnement de 7 bots dans une région ressort. Quatre bots statiques sont placés à chaque coin de la région, un autre statique va se placer au centre et deux autres vont bouger suivant un schéma défini, nous pouvons alors enregistrer les positions des avatars se trouvant dans la région et on aura sept fois les informations sur les positions et nous comparons les données pour voir si des déviations existent entre les valeurs et si elles sont importantes.\\  
 	
	Des limitations existent pour la collecte des traces, une des premières et que les mondes virtuels sont très souvent découpés en région ou en île mais les bots que nous introduisons ne peuvent pas traverser les régions et ils ne peuvent pas par exemple pas suivre des avatars. Dans {reference a rechercher}, il y aussi un problème avec la détection des avatars qui se trouvent sur des objets et il n'y pas de prise en compte de la coordonnées \textit{z}. \textbf{A CONTINUER}	

	\subsection{Observations des traces}
	 La collecte de toutes ces traces a permis de faire beaucoup d'observations sur les déplacements des avatars dans les mondes virtuels. Cela a aussi permis de faire valider ou invalider des modèles qui avaient été mis en place ( CHERCHER REF). Une des observation qui est ressortie de ces études est l'existence de différentes zones dans le monde et que les mouvements des avatars étaient très différents en fonction de la zone où ils se trouvent. Deux types de zone peuvent se dégager: des zones très peuplées avec des avatars ayant des mouvements très aléatoires et lents et des zones, qui sont entre les zones peuplées, où les avatars se déplacent rapidement et suivant souvent une trajectoire rectiligne. (VOIR SI AUTRE CHOSE) \\
	
	Ces traces ont permis aussi de détecter les habitudes des joueurs, comme le fait qu'il (finir revoir rapidement collect de traces)
	
