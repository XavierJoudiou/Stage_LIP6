\section{Pourquoi passer à des solutions pair à pair?}
	\subsection{Les solutions existantes}
	Dans la plupart des Massively Multiplayer Online Games, l'architecture est de type client/serveur. Dans cette architecture, il a forte distinction entre le client, qui envoie des requêtes au serveur et attend les réponses, et le serveur qui est à l'écoute de requêtes des clients. Cela va simplifier la sécurité et le fonctionnent global des jeux. Par exemple, pour effectuer des updates sur l'état global du jeu, il suffit de le faire sur un seule machine et il n'y aura pas de problème d'incohérence entre les données. De même pour la sécurité, toutes les données étant regroupées sur une seule machine, le contrôle sera beaucoup plus simple que dans des systèmes distribués où le nombre de point d'entrée sera beaucoup plus important. \\
	Le problème est que cette architecture ne passent pas à l'échelle, le serveur devient un goulot d'étranglement et si un trop grand nombre de joueurs se connecte, le serveur ne tiendra pas. Le problème est résolu "temporairement?" en ayant des serveurs de très grandes capacités ou en mettant en place des clusters de serveur textbf{(voir bibli)}. Mais ces solutions induisent un gros investissement dès le début de la mise en service du jeu et elles sont très chères en coût de maintenance. Un autre problème est le disponibilité du système en cas de panne du serveur, si le serveur tombe en panne alors plus personne n'aura accès à l'application que ce dernier faisait fonctionner. \\
	Au vue du nombre croissant de participant à ce genre de jeux vidéos massivement multi joueur, le passage à l'échelle devient un sujet très important et c'est pour cela que les recherches sur des architectures distribuées sont de plus en plus importantes textbf{voir si statistiques)}. \\
\newline


	\subsection{Les avantages et les inconvénients du pair à pair}
	Comme il est dit avant, l'augmentation croissante des recherches sur le sujet atteste du fait que des problématiques ressortent des solutions existantes. Le problème du passage à l'échelle est sûrement le plus important et est l'une des raisons de ces recherches. Les architectures pair à pair ne font plus ressortir d'entité serveur et client, chaque nœud sera client et serveur en fonction du moment. Les systèmes pair à pair peuvent avoir une multitude d'utilisation, que ce soit dans le partage de fichier, la communication, les jeux vidéos , le calcul scientifique, le militaire, etc. \\
	L'architecture pair à pair est telle qu'il n'y a pas de goulot d'étranglement, elle est donc plus adapté à des applications massivement multi joueur mais il faut pouvoir garantir les mêmes propriétés que les systèmes client/serveur. textbf{A REVOIR}
 

