\section{abstract}

	Depuis plusieurs années, une nouvelle classe d'application est apparue. Il s'agit des applications pair à pair, ces applications sont devenues populaires grâce à des applications de partage de fichier. De nombreuses autres utilisations de l'architecture pair à pair existent, comme pour la communication, la distribution de calculs scientifiques, les jeux vidéos multijoueur en ligne, etc. Nous allons nous intéresser aux jeux vidéos massivement multijoueur ( MMOG pour Massively Multiplayer Online Games) qui sont de plus en plus populaires et qui font ressortir des problèmes que l'architecture pair à pair doit pouvoir corriger. Le problème du passage à l'échelle sera l'un des plus importants à résoudre car des milliers de joueurs doivent pouvoir participer en même temps et avec un latence faible.\\ 
	L'architecture pair à pair est la plus adaptée pour résoudre le problème de passage à l'échelle dans les MMOGs, mais il sera plus difficile d'assurer une faible latence. Le but de ce stage est de proposer une technique efficace de prédiction du comportement des joueurs de MMOG afin de pouvoir anticiper leurs mouvements. Il est alors possible d’adapter le réseau en conséquence et ainsi diminuer la latence due au routage.\\ 
	A l'heure actuelle, les MMOGs utilisent une architecture client/serveur, ce qui pose des problèmes de passage à  l'échelle, car on aura un nombre de joueurs limités pouvant être sur chaque serveur. Ce qui fait que le jeu est découpé en plusieurs parties indépendantes, chacune étant gérée par un serveur, ce qui pose des problèmes d'unité de l'univers. Ce problème implique aussi un coût achat de serveur et en maintenance de ceux-ci très élevé.\\
	Pour remédier à cela, une solution consiste à remplacer le modèle client/serveur par un réseau logique pair à pair (overlay).Malheureusement, les protocoles pair à pair existants sont trop peu réactifs pour assurer la faible latence nécessaire à ce genre d’applications. En effet, pour une expérience de jeu satisfaisante, l’information entre deux pairs en interaction dans le MMOG doit être transmise avec une latence d’au plus quelques centaines de millisecondes. Ceci est problématique, même avec un routage efficace.\\
	Néanmoins, quelques travaux ont déjà été menés pour adresser ce problème. L’idée est d’adapter le voisinage de chaque pair afin que toute l’information dont il aura besoin dans un avenir proche se trouve à un seul hop de lui dans le réseau. Il est alors nécessaire de correctement évaluer les futurs besoins de chaque pair, et de faire évoluer son voisinage à temps.
		Le travail à réaliser est d'améliorer un overlay pair à pair pour MMOG qui a été implémenté dans le simulateur Peersim. Cet overlay anticipe les mouvements du joueur dans le MMOG et adapte le voisinage de son pair en conséquence. Cependant, les algorithmes d’anticipation sont naïfs et peu précis .
Il s’agit donc de concevoir des mécanismes efficaces d’anticipation de la trajectoire des joueurs afin de mieux adapter l’overlay à leurs déplacements et diminuer la latence.\\

\textbf{Mots Clés:} Pair à pair, Massively Multiplayer Online Games, Overlay, Movements, ...
