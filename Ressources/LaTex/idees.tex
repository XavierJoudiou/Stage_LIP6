Pourquoi passer les MMOGs vers une architecture P2P?
	
Les limites des solutions P2P actuelles.

Problèmes de sécurité, triche, l'architecture P2P est plus vulnérable aux attaques.

Mise en place d'une bonne architecture pour les MMOGs qui puissent durer longtemps et qui ne soit pas trop chère au départ et surtout en entretien. Entretien des serveurs, creations nouvelles quêtes ...

Découpage de la carte avec Voronoi ( ou autres techniques).

Études des traces des joueurs dans différents MMOGs

Principe de AOI, coordinateur, abonnements, ...

Collaboration entre les nœuds, si un nœud détecte l'arrivée d'un avatar alors prévient le voisin vers qui il se déplace. 

Distinction des différents de zones dans le mode ( populaires ou non, ...)

Expliquer DHTs, query-range DHTs,overlay basé sur les objets ( Voronet)  ...

Placements des données en fonction du déplacement de l'avatar dans les zones "vide": les deux états correspondants dans Blue Banana.

Essayer d'améliorer les mouvements lors des déplacements en groupe? Déplacements en équipe dans WOW, regarder pheromones des fourmis ? 

Regarder les objets ou les quêtes dans les AOI?

Liens entre les avatars du même groupe et proche?  

Améliorer liens en overlay et réseau physique? 
