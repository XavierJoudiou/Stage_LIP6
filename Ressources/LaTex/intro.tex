\section{Introduction}
	Dans ce rapport bibliographique, nous allons observer les différents travaux déjà réalisés sur les applications pair à pair, nous étudierons plus précisément le cas des jeux vidéos massivement multijoueur. Ces applications induisent différentes propriétés si l'on veut qu'elles soient utilisées par un grand nombre de personne en même temps et sur une longue durée. Les systèmes pair à pair vont pouvoir répondre efficacement aux différentes propriétés nécessaires au bon fonctionnement des applications. Le but du stage est d'améliorer le placement des données pour que l'utilisateur puisse avoir dans son voisinage les données qui lui seront nécessaires aux itérations suivantes, pour cela il faudra anticiper au mieux les mouvements du joueurs.\\

	Tout d'abord nous commencerons par une présentation du Laboratoire d'informatique de Paris 6, qui m'accueille pendant ce stage de fin d'étude et nous présenterons brièvement l'INRIA Pairs - Rocquencourt qui travaille conjointement avec l'équipe REGAL à laquelle je suis rattaché. Nous expliquerons ensuite pourquoi l'approche pair à pair est celle qui paraît la plus adaptée à ces applications, nous en profiterons pour rappeler rapidement les caractéristiques des différentes architectures (cf \ref{whyp2p}, page \pageref{whyp2p}). Ensuite, les mécanismes permettant une meilleure mise en place de l'architecture pair à pair seront expliquer, nous rentrerons un peu plus dans les détails pour Sollipsis car Blue Banana l'utilise comme base (cf \ref{sollipsis}, page \pageref{sollipsis}). Les différentes études des traces des avatars dans les environnements virtuels seront ensuite expliqués (cf \ref{trace}, page \pageref{trace}).Après avoir parler des différents mécanismes existants, nous parlerons alors du travail Blue Banana qu'il faudra essayer d'améliorer lors des mois restants du stage (cf \ref{BlueBanana}, page \pageref{BlueBanana}).

