\section{Introduction}
	Dans ce rapport bibliographique, nous allons observer les différents travaux déjà réalisés sur les applications pair à pair, nous étudierons plus précisément le cas des jeux vidéos massivement multijoueur. Ces applications induisent une bonne montée en charge si l'on veut qu'elles soient utilisées par un grand nombre de personne en même temps. Le but du stage est d'améliorer le placement des données pour que l'utilisateur puisse avoir dans son voisinage les données qui lui seront nécessaires ultérieurement, pour cela il faudra anticiper au mieux les mouvements du joueurs.\\
	Tout d'abord nous commencerons par une présentation du Laboratoire d'informatique de Paris 6, qui m'accueille pendant ce stage et nous présenterons brièvement l'INRIA Pairs - Rocquencourt qui travaille conjointement avec l'équipe REGAL. Nous expliquerons ensuite pourquoi l'approche pair à pair est celle qui est la plus adaptée à ces applications, nous en profiterons pour rappeler rapidement les caractéristiques des différentes architectures (cf \ref{whyp2p}, page \pageref{whyp2p}). Ensuite, les mécanismes permettant une meilleure mise en place de l'architecture pair à pair seront expliquer, nous rentrerons un peu plus dans les détails pour Sollipsis car BlueBanana s'appuie dessus(cf \ref{sollipsis}, page \pageref{sollipsis}). Nous parlerons alors du travail BlueBanana qu'il faudra essayer d'améliorer lors des mois restants du stage (cf \ref{BlueBanana}, page \pageref{BlueBanana}). A FINIR ET REVOIR

