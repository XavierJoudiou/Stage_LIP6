\section{Introduction}
	A travers ce rapport, nous allons observer les différents travaux déjà réalisés sur les applications pair à pair, nous nous intéresserons dans le détail aux jeux vidéos massivement multijoueur. Ces applications impliquent que différentes propriétés soient vérifiées si l'on veut qu'elles soient utilisées par un grand nombre de personnes en même temps et sur une longue durée. L'architecture pair à pair peut répondre efficacement à certaines des différentes propriétés nécessaires au bon fonctionnement des applications mais elle pose un problème au niveau de la latence. Le but du stage est donc d'améliorer les liens du réseau pair à pair pour que l'utilisateur puisse avoir dans son voisinage les données qui lui seront nécessaires aux itérations suivantes, pour cela il faudra anticiper au mieux les mouvements du joueur.\\

	Tout d'abord nous commencerons par une présentation du Laboratoire d'informatique de Paris 6, qui m'accueille pendant ce stage de fin d'étude et nous présenterons brièvement l'INRIA Paris - Rocquencourt qui travaille conjointement avec l'équipe REGAL à laquelle je suis rattaché. Nous expliquerons ensuite pourquoi l'approche pair à pair est celle qui paraît la plus adaptée pour répondre au différentes problématiques qu'induisent ces applications, nous en profiterons pour rappeler rapidement les caractéristiques des différentes architectures (cf \ref{whyp2p}, page \pageref{whyp2p}). Les différentes études des traces des avatars dans les environnements virtuels seront ensuite expliquées (cf \ref{trace}, page \pageref{trace}). Ensuite, les mécanismes permettant une meilleure mise en place de l'architecture pair à pair seront expliqués, nous rentrerons un peu plus dans les détails pour Solipsis car Blue Banana l'utilise comme base (cf \ref{solipsis}, page \pageref{solipsis}). Après avoir parlé des différents mécanismes déjà existants, nous parlerons alors du travail Blue Banana qu'il faudra essayer d'améliorer lors des mois restants du stage (cf \ref{BlueBanana}, page \pageref{BlueBanana}).

