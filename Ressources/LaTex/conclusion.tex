\section{Conclusion}
	Dans ce document, nous avons étudié différents mécanismes permettant d'améliorer la réactivité dans les applications pair à pair, et plus précisément dans les MMOGs. La nécessité d'amélioration des environnements virtuels ayant une architecture pair à pair est dû aux limitations de l'architecture client/serveur utilisée jusqu'à maintenant. Nous décrivons les différents mécanismes qui ont inspiré Blue Banana, qu'il faudra améliorer dans la suite du stage.\\
	Les mécanismes d'anticipation des mouvements donnent des résultats satisfaisants mais il sera nécessaire de trouver des améliorations pour mieux anticiper les mouvements en essayant de ne pas handicaper les performances de l'application. Des améliorations sur les déplacements en groupe pourrait être une piste d'amélioration des performances. Un  article, parlant des déplacements en groupe des fourmis, explique que leurs différentes observations ont servies dans l'informatique~\cite{fourmis}. 
		
