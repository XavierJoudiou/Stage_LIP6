\section{Conclusion}
	Dans ce document, nous avons étudié les différentes architectures disponibles pour les MMOGs, nous avons observé les limites et les avantages que peuvent avoir les architectures client/serveur et pair à pair. Ensuite grâce à l'étude des traces, nous avons pu dégager différents points permettant l'amélioration de la prise en compte de la mobilité dans les MMOGs. Nous avons vu différents mécanismes permettant d'améliorer la réactivité dans les applications pair à pair, et plus précisément dans les MMOGs, mais la plupart fonctionnait, au mieux, en réaction aux évènements~\cite{10.1109/SRDS.2006.33}. Blue Banana a permis de mettre en place un mécanisme d'anticipation des mouvements, ce qui permet de mieux faire évoluer le réseau. Nous essayerons dans le reste du stage d'améliorer l'anticipation des mouvements, en nous appuyant sur ce qui a été fait dans Blue Banana.\\

	\par Les mécanismes d'anticipation des mouvements donnent des résultats satisfaisants mais il sera nécessaire de trouver des améliorations pour mieux anticiper les mouvements en essayant de ne pas handicaper les performances de l'application. Les performances globales ont déjà été bien améliorer grâce aux mécanismes mis en place dans Blue Banana et cela sans trop affecté le réseau. Des améliorations sur les déplacements en groupe pourrait être une piste d'amélioration des performances. Un  article, parlant des déplacements en groupe des fourmis, explique que leurs différentes observations ont servies dans l'informatique~\cite{fourmis}. 
		
