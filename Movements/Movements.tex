\documentclass[11pt,a4paper]{article}
%\documentclass[10pt,twocolumn]{article}


\usepackage{Movements} %% cibler doc/modules/
\setlength{\columnsep}{1cm}


%\onecolumn
\begin{document}
  \fairetitre{Amélioration de la réactivité des réseaux pair à pair pour les MMOGs}{Étude bibliographique sur les déplacements en groupe dans les MMOG}{Xavier Joudiou}{Sergey Legtchenko \& Sébastien Monnet}{10/06/10}

\newpage
%\onecolumn
\tableofcontents
\newpage
%\twocolumn

\input{abstract}

\section{Introduction}
Dans ce document, nous allons étudier les déplacements en groupe des avatars dans les MMOG. Nous pourrons voir, grâce à cette étude rapide, si l'utilisation des déplacements en groupe est une piste intéressante dans l'amélioration du travail Blue Banana~\cite{191}.

\newpage
\section{"Alone Together?"}
\subsection{Le niveau d'un joueur joue un rôle important}

\subsection{Jouer en Guild}

\section{The changing Dynamic of Social Interaction in World of Warcraft}



\newpage
\bibliographystyle{plain}
\bibliography{Biblio}


 

\end{document}

