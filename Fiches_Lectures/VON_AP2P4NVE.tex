\documentclass[11pt,a4paper]{article}

\usepackage{../Ressources/LaTex/Fiche_Lecture_Type} %% cibler doc/modules/

\usepackage{fancyhdr}
\fancypagestyle{basdepage}{
\fancyfoot{} % clear all footer fields
\fancyfoot[C]{Stage Lip6 : Amélioration de la réactivité des réseaux p2p pour les MMOGs [\thepage]}
\renewcommand{\headrulewidth}{0pt}
\renewcommand{\footrulewidth}{0pt}}
\pagestyle{basdepage}

\title{Fiche de Lecture Type}
\author{Xavier Joudiou}
\date{09/04/10}

\begin{document}
  %\thispagestyle{basdepage}
  \fairetitre{Fiche de Lecture VON}{Xavier Joudiou}{22/04/10}
  \infoFicheLecture{VON: A Scalable Peer-to-Peer Network for Virtual Environments}{Shun-Yun Hu, Jui-Fa Chen, Tsu-Han Chen}{2006}{IEEE Network,20(4):22-31,Jully 2006}{P2P, Virtual Environments, MMOGs, Overlay}
	
  \begin{itemize}
  \renewcommand{\labelitemi}{$\Rightarrow$}
	\item Limites des architectures client/serveur, il y a de plus en plus d'utilisateurs et donc le passage à l'échelle devient un facteur clé. Donc on veut utiliser des systèmes pair à pair. 
	\item Il faut assurer: Consistency, Responsiveness, Scalability, Persistency, Reliability, Security. Mise en place de système de AOI ( dynamic AOI).
	\item P2P-Based NVE: le problème de la recherche de contenu est simplifié, facilité de localisation et d'identification. 
	\item VON introduit trois procédures principales:
	\begin{itemize}
		\item Join Procedure: textit{joining node} contacte le server pour avoir un ID unique. Puis envoie une requête avec ses coordonnées à tous les nœuds existants. Création de la liste des voisins, mis à jour chez les voisins, etc.
		\item Move Procedure: quand un nœud bouge ses coordonnées sont mises à jour chez ses voisins ( gestion si nœud à la limite de la région, si il y a des nouveaux voisins, etc).
		\item Leave Procedure: le nœud se déconnecte( qu'importe sa raison et la façon) et les voisins vont se mettre à jour. 
	\end{itemize}
	\item Von est similaire à DT-Overlay si seulement tous les nœuds sont juste connectés avec leurs voisins et ne bougent pas.
	\item Von permet de gérer un grand nombre d'utilisateurs et de garder un topologie consistence. Mais beaucoup de messages introduit par AOI et si les utilisateurs se déplacent trop vite, on  a des problème pour notifier les nouveaux voisins.
	\item ...
  \end{itemize}

\end{document}  
