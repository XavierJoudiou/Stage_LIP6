\documentclass[11pt,a4paper]{article}

\usepackage{../Ressources/LaTex/Fiche_Lecture_Type} %% cibler doc/modules/

\usepackage{fancyhdr}
\fancypagestyle{basdepage}{
\fancyfoot{} % clear all footer fields
\fancyfoot[C]{Stage Lip6 : Amélioration de la réactivité des réseaux p2p pour les MMOGs [\thepage]}
\renewcommand{\headrulewidth}{0pt}
\renewcommand{\footrulewidth}{0pt}}
\pagestyle{basdepage}

\title{Fiche de Lecture Type}
\author{Xavier Joudiou}
\date{09/04/10}

\begin{document}
  %\thispagestyle{basdepage}
  \fairetitre{Fiche de Lecture VON}{Xavier Joudiou}{22/04/10}
  \infoFicheLecture{VON: A Scalable Peer-to-Peer Network for Virtual Environments}{Shun-Yun Hu, Jui-Fa Chen, Tsu-Han Chen}{2006}{IEEE Network,20(4):22-31,Jully 2006}{P2P, Virtual Environments, MMOGs, Overlay}
	
  \begin{itemize}
  \renewcommand{\labelitemi}{$\Rightarrow$}
	\item Limites des architestures client/serveur, il y a de plus en plus d'utilisateurs et donc le passage à l'échelle devient un facteur clé. Donc on veut utiliser des systèmes pair à pair. 
	\item Il faut assurer: Consitency, Responsiveness, Scalabilty, Persistency, Reliability, Security. Mise en place de système de AOI.
	\item ...
  \end{itemize}

\end{document}  
