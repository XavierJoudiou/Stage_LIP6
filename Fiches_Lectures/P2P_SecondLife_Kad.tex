\documentclass[11pt,a4paper]{article}

\usepackage{../Ressources/LaTex/Fiche_Lecture_Type} %% cibler doc/modules/

\usepackage{fancyhdr}
\fancypagestyle{basdepage}{
\fancyfoot{} % clear all footer fields
\fancyfoot[C]{Stage Lip6 : Amélioration de la réactivité des réseaux p2p pour les MMOGs [\thepage]}
\renewcommand{\headrulewidth}{0pt}
\renewcommand{\footrulewidth}{0pt}}
\pagestyle{basdepage}

\title{Fiche de Lecture Type}
\author{Xavier Joudiou}
\date{14/04/10}

\begin{document}
  %\thispagestyle{basdepage}
  \fairetitre{Fiche de Lecture P2P Second Life}{Xavier Joudiou}{14/04/10}
  \infoFicheLecture{P2P Second Life: experimental validation using Kad}{Matteo Varvello, Christophe Diot, Ernst Biersack Thomson}{Avril 2009}{In Infocom 2009,28th IEEE Conference on Computer Communications,pages 19-25, Rio De Janeiro, Brazil}{P2P, Second Life, Social Virtual Worlds}
	
  \begin{itemize}
  \renewcommand{\labelitemi}{$\Rightarrow$}
	\item Beaucoup d'utilisateurs donc déploiement qui doit tenir de fortes charges,avoir une vue consistence du monde pour tous les joueurs, ne pas avoir de perte d'objet. Aujourd'hui SVWs sont en client/serveur.
	\item Découpage du monde de SL en région qui sont chacune associées à un serveur dédié. Limitation du nombre d'avatar par région.
	\item P2P permet de dépenser moins pour jouer à plus !!!
	\item Utilisation de Kad (eMule), monde dynamique, AOI, s'abonner a des coordinateurs pour connaitre modification faite sur un objet dans son AOI, pour ajouter un objet création d'une clé en fonction des coordonnées et on informe le coordinateur de la cellule de l'objet.
	\item Nombre d'objet dans une cellule, si $\ge$ à Dmax on divise en deux la cellule, de même on peut faire des merge si < à Dmin. Un objet "pointer" est introduit pour informer les utilisateurs si merge ou split. Pour déplacement contacter le coordinateur à chaque fois.
	\item Inconsistence de la vue du monde quand avatar arrive, quand il bouge à travers la cellule. Le premier point peut être réglé facilement ( mettre en cache à travers les différentes sessions    
	\item ...
  \end{itemize}

\end{document}  
