\documentclass[11pt,a4paper]{article}

\usepackage{../Ressources/LaTex/Fiche_Lecture_Type} %% cibler doc/modules/

\usepackage{fancyhdr}
\fancypagestyle{basdepage}{
\fancyfoot{} % clear all footer fields
\fancyfoot[C]{Stage Lip6 : Amélioration de la réactivité des réseaux p2p pour les MMOGs [\thepage]}
\renewcommand{\headrulewidth}{0pt}
\renewcommand{\footrulewidth}{0pt}}
\pagestyle{basdepage}

\title{Fiche de Lecture Type}
\author{Xavier Joudiou}
\date{09/04/10}

\begin{document}
  %\thispagestyle{basdepage}
  \fairetitre{Fiche de Lecture Scalable P2P Networked Virtual Environment}{Xavier Joudiou}{27/04/10}
  \infoFicheLecture{Scalable Peer-to-Peer Networked Virtual Environment}{Shun-Yun Hu, Guna-Ming Liao}{2004}{Proceedings of 3rd ACM SIGCOMM workshop on Network and system support for games}{NVE, P2P, massively multiplayer, Voronoi diagram, scalability, interest management}
	
  \begin{itemize}
  \renewcommand{\labelitemi}{$\Rightarrow$}
	\item Les applications d'environnement virtuel en réseau ont commencées à apparaître il y a longtemps(début de SIMNET 1980) que ce soit dans le militaire, dans les jeux vidéos, ou des films. Aujourd'hui, ces applications sont en pleine expansion.
	\item Il faut régler les questions suivantes pour avoir une applications la plus intéressante possible: Consistency, Performance/Responsiveness, Security, textbf{Scalability}, Persistency, Reliability/Fault-tolerance. 
	\item Mise en place d'une solution Pair à pair construite avec le diagramme de Voronoi, et avec un seul serveur léger qui servira de point d'accès et qui permettra l'authentification. textbf{Pas de prise de le sécurité et de la persistence dans leur solution}. Introduction de l'idée de AOI. Découpage de la carte en zone et mise en place d'un coordinateur par zone.
	\item Explications des différentes procedures de connection, déconnexion, déplacements, etc. Chaque nœud connait les états de ses voisins.
	\item Ils disent que la sécurité sera un frein au développement des jeux totalement en P2P.
	\item ...
  \end{itemize}

\end{document}  
