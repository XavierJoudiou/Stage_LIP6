\documentclass[11pt,a4paper]{article}

\usepackage{Fiche_Lecture_Type} %% cibler doc/modules/

\usepackage{fancyhdr}
\fancypagestyle{basdepage}{
\fancyfoot{} % clear all footer fields
\fancyfoot[C]{Stage Lip6 : Amélioration de la réactivité des réseaux p2p pour les MMOGs [\thepage]}
\renewcommand{\headrulewidth}{0pt}
\renewcommand{\footrulewidth}{0pt}}
\pagestyle{basdepage}

\title{Fiche de Lecture Type}
\author{Xavier Joudiou}
\date{08/04/10}

\begin{document}
  %\thispagestyle{basdepage}
  \fairetitre{Fiche de Lecture Thèse Gwendal Simon}{Xavier Joudiou}{08/04/10}
  \infoFicheLecture{Conception et réalisation d'un système pour environnements virtuels massivement partagés}{Gwendal Simon}{10 décembre 2004}{Thèse, équipe ADEPT}{Poste-à-poste, Réalité virtuelle partagée, Réseaux ad-hoc, Géométrie Algorithmique,Traitement réparti}
	
  \begin{itemize}
  \renewcommand{\labelitemi}{$\Rightarrow$}
	\item Collaboration des pairs voisins pour indiquer, si une entité C se déplace vers B et que A détecte alors A prévient B. Collaboration dans les deux sens .
	\item Introduction de la notion de nœud : identifiant, pseudo, calibre, nombres de voisins espérés, position, orientation et rayon de connaissance
	\item Détection des arrivées, des départs, changement de propriété ...
	\item Volonté d'une architecture flexible et indépendant des langages ou des OS.:
	\item ...
  \end{itemize}

\end{document}  
