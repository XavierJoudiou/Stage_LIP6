\documentclass[11pt,a4paper]{article}

\usepackage{../Ressources/LaTex/Fiche_Lecture_Type} %% cibler doc/modules/

\usepackage{fancyhdr}
\fancypagestyle{basdepage}{
\fancyfoot{} % clear all footer fields
\fancyfoot[C]{Stage Lip6 : Amélioration de la réactivité des réseaux p2p pour les MMOGs [\thepage]}
\renewcommand{\headrulewidth}{0pt}
\renewcommand{\footrulewidth}{0pt}}
\pagestyle{basdepage}

\title{Fiche de Lecture Type}
\author{Xavier Joudiou}
\date{07/04/10}

\begin{document}
  %\thispagestyle{basdepage}
  \fairetitre{Fiche de Lecture Solipsis MMPVW}{Xavier Joudiou}{07/04/10}
  \infoFicheLecture{Solipsis : A Massively Multi-Participant Virtual World}{Joaquin Keller, Gwendal Simon}{Juin 2003}{CSREA Press}{Peer To Peer System, Shared Virtual Reality, Massively Distributed Algorithms,Self-organizing Systems, Computational Geometry}
	
  \begin{itemize}
  \renewcommand{\labelitemi}{$\Rightarrow$}
	\item Deux entités doivent avoir la même vision, intérêt croissant pour les applications de réalité virtuelle partagée. Problèmes de passage à l'échelle ( cout des serveurs ...).
	\item Collaboration spontanée : une entité indique a un de ces voisins qu'une autre entité vient vers elle, collaboration aussi requête/réponse pour verifier qu'on se situe bien dans l'enveloppe convexe des adjacent de e.
	\item Connexion et téléportation : une entité e qui se connecte ou se téléporte dans le monde virtuel, doit connaitre une entité e0 connectée et sa future position posC . Algorithme de localisation inverse , tourne autour de posC pour trouve l'entité la plus proche
	\item Garantit la connaissance des plus proches voisins, pas d'iles, maintien des prop. Si mobilité,
	\item Peut gérer des millions d'entités dans le système.
	\item ...
  \end{itemize}

\end{document}  
