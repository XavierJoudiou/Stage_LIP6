\documentclass[11pt,a4paper]{article}

\usepackage{Fiche_Lecture_Type} %% cibler doc/modules/

\usepackage{fancyhdr}
\fancypagestyle{basdepage}{
\fancyfoot{} % clear all footer fields
\fancyfoot[C]{Stage Lip6 : Amélioration de la réactivité des réseaux p2p pour les MMOGs [\thepage]}
\renewcommand{\headrulewidth}{0pt}
\renewcommand{\footrulewidth}{0pt}}
\pagestyle{basdepage}

\title{Fiche de Lecture Type}
\author{Xavier Joudiou}
\date{07/04/10}

\begin{document}
  %\thispagestyle{basdepage}
  \fairetitre{Fiche de Lecture Solipsis MMPVW}{Xavier Joudiou}{07/04/10}
  \infoFicheLecture{Solipsis : A Massively Multi-Participant Virtual World}{Joaquin Keller, Gwendal Simon}{Juin 2003}{CSREA Press}{Peer To Peer System, Shared Virtual Reality, Massively Distributed Algorithms,Self-organizing Systems, Computational Geometry}
	
  \begin{itemize}
  \renewcommand{\labelitemi}{$\Rightarrow$}
	\item Deux entités doivent avoir la meme vision, interet croissant pour les applications de réalité virtuelle partagée. Problème de passage à l'echelle ( cout des serveurs ...)
	\item Collaboration spontannée : une entite indique a un de ces voisins q'une autre entite vient vers elle, collaboration aussi requete/reponse pour verifier qu on se situe bien dans l'enveloppe convexe des adjacent de e.
	\item Connexion et teleportation : une entite e qui se connecte ou se teleporte dans le monde virtuel, doit connaitre une entite e0 connectee et sa future position posC . Algorithme de localisation inverse , tourne autour de posC pour trouve l'entite la plus proche
	\item Garantit la connaissance des plus proches voisins, pas d'iles, maintien des prop. si mobilité,
	\item Peut gerer des millions d'entites dans le systeme,
	\item ...
  \end{itemize}

\end{document}  
