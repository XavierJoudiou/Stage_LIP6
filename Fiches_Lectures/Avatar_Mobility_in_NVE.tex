\documentclass[11pt,a4paper]{article}

\usepackage{../Ressources/LaTex/Fiche_Lecture_Type} %% cibler doc/modules/

\usepackage{fancyhdr}
\fancypagestyle{basdepage}{
\fancyfoot{} % clear all footer fields
\fancyfoot[C]{Stage Lip6 : Amélioration de la réactivité des réseaux p2p pour les MMOGs [\thepage]}
\renewcommand{\headrulewidth}{0pt}
\renewcommand{\footrulewidth}{0pt}}
\pagestyle{basdepage}

\title{Fiche de Lecture Type}
\author{Xavier Joudiou}
\date{09/04/10}

\begin{document}
  %\thispagestyle{basdepage}
  \fairetitre{Fiche de Lecture Avatar Mobility in NVE}{Xavier Joudiou}{12/04/10}
  \infoFicheLecture{Avatar Mobility in Networked Virtual Environments: Measurements, Analysis and Implications}{Huiguang Liang, Ina Tay, Ming Feng Neo, Wei Tsang Ooi, Mehul Motani}{15 juillet 2008}{}{Networked Virtual Environnement, Avatar Mobility, Avatar Behavior, Caching, Interest Management, Load Balancing, Peer-to-Peer}
	
  \begin{itemize}
  \renewcommand{\labelitemi}{$\Rightarrow$}
	\item Pas assez de système pour tester, dépendance envers des avatars ...
	\item Récupérer pleins de traces de joueurs dans Second Life en utilisant un bot qui collecte des informations à intervalle régulier, pour analyser les déplacements des avatars, vérifier les modèles ...
	\item Différences entre WOW et Second Life, l'environnement de WOW est beaucoup plus statique alors que dans SL on peut créer, bouger et modifier des objets.
	\item Limitation de la collecte des traces des avatars: les bots ne peuvent pas traverser les régions et il n'y a pas de prises en compte de la coordonnées z. Il y aussi un problème pour détecter les positions des avatars quand ceux-ci sont assis sur des objets.
	\item Les avatars reviennent dans la même zone durant une partie,on a moyenne de retour qui est entre 45 et 60 minutes,ils font une étude sur le temps resté dans une region, un avatar qui reste longtemps dans un même zone  pourrait être des supers nœuds.
	\item Les jeux tels que Second Life ne sont pas très scalable, les joueurs bougent vite dans les regions "vides", Second Life prefetch les données autour d'un avatar (zone circulaire) car 18 \% tourne autour de leur position.
	\item Il y a des regroupement d'avatar dans les zones d'intérêt et les déplacements, dans ces zones, se font à une vitesse réduite.
	\item ...
  \end{itemize}

\end{document}  
