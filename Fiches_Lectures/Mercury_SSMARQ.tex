\documentclass[11pt,a4paper]{article}

\usepackage{../Ressources/LaTex/Fiche_Lecture_Type} %% cibler doc/modules/

\usepackage{fancyhdr}
\fancypagestyle{basdepage}{
\fancyfoot{} % clear all footer fields
\fancyfoot[C]{Stage Lip6 : Amélioration de la réactivité des réseaux p2p pour les MMOGs [\thepage]}
\renewcommand{\headrulewidth}{0pt}
\renewcommand{\footrulewidth}{0pt}}
\pagestyle{basdepage}

\title{Fiche de Lecture Type}
\author{Xavier Joudiou}
\date{09/04/10}

\begin{document}
  %\thispagestyle{basdepage}
  \fairetitre{Fiche de Lecture Mercury}{Xavier Joudiou}{22/04/10}
  \infoFicheLecture{Mercury: Supporting Scalable Multi-Attribute Range Queries}{Ashwin R.Bharambe, Mukesh Agrawal, Srinivasan Seshan}{2004}{SIGCOMM'04, Aug. 30-Sept. 3,2004,Portland, Oregon, USA}{Range queries, peer-to-peer systems, distributed hash tables, laod balancing, random sampling}
	
  \begin{itemize}
  \renewcommand{\labelitemi}{$\Rightarrow$}
	\item multi-attribute range queries by creating a routing hub fot each attibute in the application schema. Mercury oganizes each routing hub into a circular overlay of nodes and places data continuously on this ring, each node is responsible for a rang of values for the particular attribute. Les données ne sont pas placées aléatoirement. 
	\item Dans Mercury, les données sont représentées comme une liste de paire attibut-valeur (type, attribute, value). conjunction $\rightarrow$ (type, attribute, operator, value)
	\item a finir le we du 24/04 
	\item ...
  \end{itemize}

\end{document}  
