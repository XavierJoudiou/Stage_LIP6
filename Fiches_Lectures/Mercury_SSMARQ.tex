\documentclass[11pt,a4paper]{article}

\usepackage{../Ressources/LaTex/Fiche_Lecture_Type} %% cibler doc/modules/

\usepackage{fancyhdr}
\fancypagestyle{basdepage}{
\fancyfoot{} % clear all footer fields
\fancyfoot[C]{Stage Lip6 : Amélioration de la réactivité des réseaux p2p pour les MMOGs [\thepage]}
\renewcommand{\headrulewidth}{0pt}
\renewcommand{\footrulewidth}{0pt}}
\pagestyle{basdepage}

\title{Fiche de Lecture Type}
\author{Xavier Joudiou}
\date{09/04/10}

\begin{document}
  %\thispagestyle{basdepage}
  \fairetitre{Fiche de Lecture Mercury}{Xavier Joudiou}{22/04/10}
  \infoFicheLecture{Mercury: Supporting Scalable Multi Attribute Range Queries}{Ashwin R.Bharambe, Mukesh Agrawal, Srinivasan Seshan}{2004}{SIGCOMM'04, Aug. 30-Sept. 3,2004,Portland, Oregon, USA}{Range queries, peer-to-peer systems, distributed hash tables, load balancing, random sampling}
	
  \begin{itemize}
  \renewcommand{\labelitemi}{$\Rightarrow$}
	\item multi-attribute range queries by creating a routing hub for each attribute in the application schema. Mercury organizes each routing hub into a circular overlay of nodes and places data continuously on this ring, each node is responsible for a rang of values for the particular attribute. Les données ne sont pas placées aléatoirement. 
	\item Mercury routing protocol est inspiré de Chord, Symphony, etc. Dans Mercury, les données sont représentées comme une liste de paire attibut-valeur (type, attribute, value). Conjunction $\rightarrow$ (type, attribute, operator, value). 
	\item Like Symphony, the key to Mercury's route optimization is the selection of \textit{k long distance} links that are maintained in addition to the successor and predecessor links.	
	\item textbf{Node Join}: Le nœud doit déjà avoir des informations sur au moins un nœud du système pour se connecter, il interroge un nœud et obtient des informations sur les "hubs", il choisit ensuite un "hub" aléatoirement et contacte un des nœuds. Il copie ensuite le "routing state" de son successeur et initialise deux processus de maintenance.
	\item \textbf{Node Departure}: Il faut "refaire" les liens entre le successeur et le prédécesseur, le lien de longue distance et les liens inter-hub.Pour le lien successeur/prédécesseur, chaque nœud a une courte liste pour "réparer" le lien. Pour le lien de longue distance, création d'un nouveau lien et périodiquement reconstruction des liens de longue distance en utilisant une estimation récente du nombre de nœud. Pour les liens inter-hub, il y a trois choix, premièrement utilisation d'une sauvegarde de "cross-hub link", si pas possible demande à son successeur et prédécesseur et sinon demande au "bootstrap server".
	\item Mercury, building block for distributed applications ( Online Games, etc.). Une des difficultés dans les MMOGs est de maintenir le "Game State",le challenge est de fournir une manière pour les mises à jours de la vue du "Game State" des nœuds. Idée de AOI et public/subcr pour Maj des états.
	\item API pour pub/sub, trois appel basique : send\_publication(envoyer un objet via Mercury), register\_interest(s'abonner au Maj), unregister\_interest(se désabonner).
	\item ...
  \end{itemize}

\end{document}  
