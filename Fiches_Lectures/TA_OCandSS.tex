\documentclass[11pt,a4paper]{article}

\usepackage{Fiche_Lecture_Type} %% cibler doc/modules/

\usepackage{fancyhdr}
\fancypagestyle{basdepage}{
\fancyfoot{} % clear all footer fields
\fancyfoot[C]{Stage Lip6 : Amélioration de la réactivité des réseaux p2p pour les MMOGs [\thepage]}
\renewcommand{\headrulewidth}{0pt}
\renewcommand{\footrulewidth}{0pt}}
\pagestyle{basdepage}

\title{Fiche de Lecture Type}
\author{Xavier Joudiou}
\date{09/04/10}

\begin{document}
  %\thispagestyle{basdepage}
  \fairetitre{Fiche de Lecture T-A OC and SS}{Xavier Joudiou}{12/04/10}
  \infoFicheLecture{Topologically-Aware Overlay Construction and Server Selection}{Sylvia Ratnasamy, Mark Handley, Richard Karp, Scott Shenker}{2002}{Présentaion de Dekar Lyes}{Overlay networks, serveur}
	
  \begin{itemize}
  \renewcommand{\labelitemi}{$\Rightarrow$}
	\item routage se fait au niveau applicatif, et peut etre contradictoire avec le routage niveau IP, ignore le reseau physique, grosses latences et traffic inutile
	\item faire un routage en phase avec le niveau physique serait mieux, amélioration du tps de réponse en choisissant le serveur le plus proche du client
	\item Grande connaissance de la topologie pour des grands réseau, DIFFICILE !!!
	\item Avoir une connaissance même approximative de la topologie
	\item suite pas interessante ??
	\item ...
  \end{itemize}

\end{document}  
