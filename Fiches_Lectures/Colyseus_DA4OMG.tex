\documentclass[11pt,a4paper]{article}

\usepackage{../Ressources/LaTex/Fiche_Lecture_Type} %% cibler doc/modules/

\usepackage{fancyhdr}
\fancypagestyle{basdepage}{
\fancyfoot{} % clear all footer fields
\fancyfoot[C]{Stage Lip6 : Amélioration de la réactivité des réseaux p2p pour les MMOGs [\thepage]}
\renewcommand{\headrulewidth}{0pt}
\renewcommand{\footrulewidth}{0pt}}
\pagestyle{basdepage}

\title{Fiche de Lecture Type}
\author{Xavier Joudiou}
\date{09/04/10}

\begin{document}
  %\thispagestyle{basdepage}
  \fairetitre{Fiche de Lecture Colyseus : a Dist. Arch. for OMG}{Xavier Joudiou}{16/04/10}
  \infoFicheLecture{Colyseus: A Distributed Architecure for Online Multiplayer Games}{Ashwin Bharambe, Jeffrey Pang, Srinivasan Seshan}{2006}{Proceedings of the 3rd conference on Networked Systems Design and Implementation - Volume 3{Online Multiplayer Games, distributed architecture}
	
  \begin{itemize}
  \renewcommand{\labelitemi}{$\Rightarrow$}
	\item Jeux en réseau souvent une approche client/serveur, joueur envoi info au serveur qui revoit aux autres joueurs. Problème de scalabilité et de robustesse.
	\item Observation de popular "waypoint"  
	\item ...
  \end{itemize}

\end{document}  
