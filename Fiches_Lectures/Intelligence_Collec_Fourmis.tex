\documentclass[11pt,a4paper]{article}

\usepackage{../Ressources/LaTex/Fiche_Lecture_Type} %% cibler doc/modules/

\usepackage{fancyhdr}
\fancypagestyle{basdepage}{
\fancyfoot{} % clear all footer fields
\fancyfoot[C]{Stage Lip6 : Amélioration de la réactivité des réseaux p2p pour les MMOGs [\thepage]}
\renewcommand{\headrulewidth}{0pt}
\renewcommand{\footrulewidth}{0pt}}
\pagestyle{basdepage}

\title{Fiche de Lecture Type}
\author{Xavier Joudiou}
\date{09/04/10}

\begin{document}
  %\thispagestyle{basdepage}
  \fairetitre{Fiche de Lecture Intelligence collective des fourmis et des nouvelles techniques d'optimisation}{Xavier Joudiou}{27/04/10}
  \infoFicheLecture{Intelligence collective des fourmis et des nouvelles techniques d'optimisation}{Guy Theraulaz, Eric Bonabeau, Marco Dorigo}{2000}{CNRS INFO Numéro 386, septembre 2000:29-3O}{Intelligence collective}
	
  \begin{itemize}
  \renewcommand{\labelitemi}{$\Rightarrow$}
	\item La recherche sur les comportements sociaux des fourmis intéresse beaucoup de domaines comme l'informatique et l'ingénierie. 
	\item textit{Intelligence en essaim}, des algorithmes en sont sortis "optimisation par colonie de fourmis" et "routage par colonie de fourmis". 
	\item Pheromones pour le déplacement en groupe, et elles suivront quasiment le même chemin.
	\item ...
  \end{itemize}

\end{document}  
