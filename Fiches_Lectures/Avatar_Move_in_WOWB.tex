\documentclass[11pt,a4paper]{article}

\usepackage{../Ressources/LaTex/Fiche_Lecture_Type} %% cibler doc/modules/

\usepackage{fancyhdr}
\fancypagestyle{basdepage}{
\fancyfoot{} % clear all footer fields
\fancyfoot[C]{Stage Lip6 : Amélioration de la réactivité des réseaux p2p pour les MMOGs [\thepage]}
\renewcommand{\headrulewidth}{0pt}
\renewcommand{\footrulewidth}{0pt}}
\pagestyle{basdepage}

\title{Fiche de Lecture Type}
\author{Xavier Joudiou}
\date{09/04/10}

\begin{document}
  %\thispagestyle{basdepage}
  \fairetitre{Fiche de Lecture Avatar Movement in WOWB}{Xavier Joudiou}{15/04/10}
  \infoFicheLecture{Avatar Movement in Worlds of Warcraft Battlegrounds}{John L.Miller Jon Crowcroft}{2009}{NetGames 2009}{Avatar Movement, WOW}
	
  \begin{itemize}
  \renewcommand{\labelitemi}{$\Rightarrow$}
	\item Organisation en deux factions, présence de hotspots ou les joueurs passent la plupart du temps, Déplacement suivant un chemin bien défini entre les destinations, Déplacement en groupes. 
	\item Hotspots souvent facile à localiser ( drapuex, cimetières ...)
	\item Waypoint Navigation: chemin "fixé", contraintes géographique du terrain, "diffèrents types de chemins pour type de joueurs".
	\item Mouvement groupé: avatar proche les uns des autres.
	\item Problème trouvé d'autres solutions pour améliorer car ces derniers points sont variables en fonctions des avatars.
	\item ...
  \end{itemize}

\end{document}  
