\documentclass[11pt,a4paper]{article}

\usepackage{../Ressources/LaTex/Fiche_Lecture_Type} %% cibler doc/modules/

\usepackage{fancyhdr}
\fancypagestyle{basdepage}{
\fancyfoot{} % clear all footer fields
\fancyfoot[C]{Stage Lip6 : Amélioration de la réactivité des réseaux p2p pour les MMOGs [\thepage]}
\renewcommand{\headrulewidth}{0pt}
\renewcommand{\footrulewidth}{0pt}}
\pagestyle{basdepage}

\title{Fiche de Lecture Type}
\author{Xavier Joudiou}
\date{09/04/10}

\begin{document}
  %\thispagestyle{basdepage}
  \fairetitre{Fiche de Lecture Voronet}{Xavier Joudiou}{06/04/10}
  \infoFicheLecture{VoroNet: A scalable object network based on Voronoi tessellations}{Olivier Beaumont , Anne-Marie Kermarrec , Loris Marchal , Étienne Riviere}{Février 2006}{Research Report No RR2006-11}{Overlay network, peer-to-peer, scalability, small-world networks, Voronoi
                    tessellations, Computational geometry.}
	
  \begin{itemize}
  \renewcommand{\labelitemi}{$\Rightarrow$}
	\item bcp de reseau p2p avec DHTs, Voronet => objec-based overlay network, lien entre les objets au lieu des noeuds physiques. Chaque noeud est responsable des objects qui la publier.
	\item voisin proche et voisin lointain, creation et suppression de region de Voronoi
	\item nombre d'objets connus => extension dynamique possible en recherche
	\item probleme avec nombre de voisins non bornée ? 
	\item ...
  \end{itemize}

\end{document}  
