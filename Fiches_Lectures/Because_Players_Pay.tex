\documentclass[11pt,a4paper]{article}

\usepackage{../Ressources/LaTex/Fiche_Lecture_Type} %% cibler doc/modules/

\usepackage{fancyhdr}
\fancypagestyle{basdepage}{
\fancyfoot{} % clear all footer fields
\fancyfoot[C]{Stage Lip6 : Amélioration de la réactivité des réseaux p2p pour les MMOGs [\thepage]}
\renewcommand{\headrulewidth}{0pt}
\renewcommand{\footrulewidth}{0pt}}
\pagestyle{basdepage}

\begin{document}
  %\thispagestyle{basdepage}
  \fairetitre{Fiche de Lecture Because Players Pay}{Xavier Joudiou}{09/04/10}
  \infoFicheLecture{Because Players Pay : The Business Model Influence on MMOG Design}{Tiago Reis Alves, Licinio Roque}{2007}{DiGRA 2007 Conference}{MMOG, Game Design, Business Model, Actor-Network Theory, Massively Multiplayer Online Games}
	
  \begin{itemize}
  \renewcommand{\labelitemi}{$\Rightarrow$}
	\item Les jeux doivent avoir une longue durée de vie pour engranger le plus d'argent, beaucoup de joueurs doivent s'inscrire,il y a donc beaucoup de joueurs à gérer en même temps.
	\item Il  faut adapter le modèle économique en fonction du jeu. Pour les jeux où il y a une partie jeu en réseau, il faut de la maintenance. Le  modèle économique est différent des jeux "normaux" où il n'y a pas de suivi après la vente du produit à la différence des jeux en ligne où il faut garder le jeu fonctionnel, donc plus de dépenses pour les serveurs , la maintenance etc.
	\item Il ya a différents types de manière de rentabiliser le jeux ( abonnement, achat virtuel, pub ...), le jeu peut être gratuit puis payant, les jeux peuvent être gratuit et payant sur abonnement après une période d'essai par exemple.
	\item Même si ce n'est pas un coté informatique, il y a des répercussions sur le développement car maintenir pour maintenir un jeu, il faut maintenir les serveurs, ajouter des nouvelles quêtes, etc. Il faut prévoir en amont. Investissement futurs ...
	\item ...
  \end{itemize}

\end{document}  
