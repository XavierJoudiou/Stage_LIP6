\documentclass[11pt,a4paper]{article}

\usepackage{/home/xavier/Stage_LIP6/Ressources/LaTex/Fiche_Lecture_Type} %% cibler doc/modules/

\usepackage{fancyhdr}
\fancypagestyle{basdepage}{
\fancyfoot{} % clear all footer fields
\fancyfoot[C]{Stage Lip6 : Amélioration de la réactivité des réseaux p2p pour les MMOGs [\thepage]}
\renewcommand{\headrulewidth}{0pt}
\renewcommand{\footrulewidth}{0pt}}
\pagestyle{basdepage}

\begin{document}
  %\thispagestyle{basdepage}
  \fairetitre{Fiche de Lecture Because Players Pay}{Xavier Joudiou}{09/04/10}
  \infoFicheLecture{Because Players Pay : The Business Model Influence on MMOG Design}{Tiago Reis Alves, Licinio Roque}{2007}{DiGRA 2007 Conference}{MMOG, Game Design, Business Model, Actor-Network Theory, Massively Multiplayer Online Games}
	
  \begin{itemize}
  \renewcommand{\labelitemi}{$\Rightarrow$}
	\item Jeux qui doivent durer longtemps, bcp de joueurs doivent s'incrire, bcp de joueurs à gérer en même temps,
	\item Adapter le modèle économique en fonction du jeu, de la maintenance nécessaire, modéle économique différent des jeux "normaux" où il n'y a pas d e suivi après la vente du produit car jeux en ligne donc coût pour garder le jeu fonctionnel, pour les serveurs ...,
	\item Différents type de manière de rentabiliser le jeux ( abo, achat virtuel, pub ...), gratuit puis payant , jeux peuvent etre gratuit et payant sur abonnement apres une periode d'essai par exemple.
	\item Meme si ce cote pas informatique, il a des repercusions sur le développement car maintenir un jeux, serveurs, quetes, ... c'est un à prévoir en amont. Investissement futurs ...
	\item ...
  \end{itemize}

\end{document}  
