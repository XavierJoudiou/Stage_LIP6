\documentclass[11pt,a4paper]{article}

\usepackage{../Ressources/LaTex/Fiche_Lecture_Type} %% cibler doc/modules/

\usepackage{fancyhdr}
\fancypagestyle{basdepage}{
\fancyfoot{} % clear all footer fields
\fancyfoot[C]{Stage Lip6 : Amélioration de la réactivité des réseaux p2p pour les MMOGs [\thepage]}
\renewcommand{\headrulewidth}{0pt}
\renewcommand{\footrulewidth}{0pt}}
\pagestyle{basdepage}

\title{Fiche de Lecture Type}
\author{Xavier Joudiou}
\date{09/04/10}

\begin{document}
  %\thispagestyle{basdepage}
  \fairetitre{Fiche de Lecture Exploiting semantic proximity in P2P content searching}{Xavier Joudiou}{16/04/10}
  \infoFicheLecture{Exploiting semantic proximity in peer-to-peer content searching}{Spyros Voulgaris, Anne-Marie Kermarrec, Laurent Massoulié, Maarten van Steen}{Mai 2004}{In 10th International Workshop on Future Trends in Distributed Computing Systems (FTDCS 2004)}{Peer-to-peer, overlay}
	
  \begin{itemize}
  \renewcommand{\labelitemi}{$\Rightarrow$}
	\item file sharing systems, clustering nodes with similar interests.
	\item candidate strategies: (stratégies pour la recherche) 
	\begin{itemize}
		\item LRU: contient les éléments les plus récemment utilisés
		\item History: on maintient les liens sémanthiques vers la même classe de préférence, cette technique obligé le maintien d'un noeud "counter". Technique lourde en nombre de messages et en stockage, et problème avec les "counters" dans certains cas.  
		\item Popularity: création de liens antre les noeuds du même type, introduction de deux paramètres: Numrep ( nombre de réponse positive pour obtenir le document) et Lastreply ( date de la dernière réponse )
	\end{itemize} 
	\item Popularity est l'algorithme le plus adapté, elle est efficace et pas trop complexe.
	\item ...
  \end{itemize}

\end{document}  
