\documentclass[11pt,a4paper]{article}

\usepackage{../Ressources/LaTex/Fiche_Lecture_Type} %% cibler doc/modules/

\usepackage{fancyhdr}
\fancypagestyle{basdepage}{
\fancyfoot{} % clear all footer fields
\fancyfoot[C]{Stage Lip6 : Amélioration de la réactivité des réseaux p2p pour les MMOGs [\thepage]}
\renewcommand{\headrulewidth}{0pt}
\renewcommand{\footrulewidth}{0pt}}
\pagestyle{basdepage}

\title{Fiche de Lecture Type}
\author{Xavier Joudiou}
\date{09/04/10}

\begin{document}
  %\thispagestyle{basdepage}
  \fairetitre{Fiche de Lecture Distributed Avatar Management for Second Life}{Xavier Joudiou}{15/04/10}
  \infoFicheLecture{Distributed Avatar Management for Second Life}{Matteo Varvello, Stefano Ferrari, Ernst Biersack, Christophe Diot}{23 novembre 2009}{NetGames 2009}{Second Life, Distributed Avatar Management, P2P, Delaunay Network}
	
  \begin{itemize}
  \renewcommand{\labelitemi}{$\Rightarrow$}
	\item Second Life: bcp de joueurs, monde divisé en région, une région par serveur, client/serveur.
	\item Client/serveur $\Rightarrow$ pas scalable, maximum 100 pers/serveur, ralentissements fréquent.
	\item P2P regle la scalabilte ?
	\item Utilisation de la triangularisation de Delaunay
	\item reste de la présentation pas très expliqué.
	\item ...
  \end{itemize}

\end{document}  
