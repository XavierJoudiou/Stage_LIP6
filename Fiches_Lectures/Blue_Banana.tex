\documentclass[11pt,a4paper]{article}

\usepackage{../Ressources/LaTex/Fiche_Lecture_Type} %% cibler doc/modules/

\usepackage{fancyhdr}
\fancypagestyle{basdepage}{
\fancyfoot{} % clear all footer fields
\fancyfoot[C]{Stage Lip6 : Amélioration de la réactivité des réseaux p2p pour les MMOGs [\thepage]}
\renewcommand{\headrulewidth}{0pt}
\renewcommand{\footrulewidth}{0pt}}
\pagestyle{basdepage}

\title{Fiche de Lecture Type}
\author{Xavier Joudiou}
\date{09/04/10}

\begin{document}
  %\thispagestyle{basdepage}
  \fairetitre{Fiche de Lecture Blue Banana }{Xavier Joudiou}{08/04/10}
  \infoFicheLecture{Blue Banana: resilience to avatar mobility in distributed MMOGs}{Sergey Legtchenko, Sébastien Monnet , Gaël Thomas}{Décembre 2009}{Rapport de recherche N°7149, INRIA}{Resilience, Mobility, Adaptability, Massively Multiplayer Online Games, Overlay}
	
  \begin{itemize}
  \renewcommand{\labelitemi}{$\Rightarrow$}
	\item MMOGs tres souvent en client serveur donc chere a maintenir et pas tres scalable
	\item Utilisation du architecture pair à pair, anticipation des mouvement des avatars, deux etats possibles pour l'avatar soir T(travelling) ou E(Exploring) car observations avec traces des joueurs ( entre les hotspots et dans les hotspots )
	\item Si en etat T, on utilise BlueBanana et si en etat E non. Algo au dessus de Solipsis.
	\item Solipsis, overlay design, chaque noeud est responsable d'un avatar, collaboration des noeuds .
	\item Essayer d'anticiper mieux les mouvements relatifs antre deux avatars, indépendament de leur position.
	\item ...
  \end{itemize}

\end{document}  
