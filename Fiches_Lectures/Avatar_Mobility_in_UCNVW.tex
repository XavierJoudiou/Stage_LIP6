\documentclass[11pt,a4paper]{article}

\usepackage{../Ressources/LaTex/Fiche_Lecture_Type} %% cibler doc/modules/

\usepackage{fancyhdr}
\fancypagestyle{basdepage}{
\fancyfoot{} % clear all footer fields
\fancyfoot[C]{Stage Lip6 : Amélioration de la réactivité des réseaux p2p pour les MMOGs [\thepage]}
\renewcommand{\headrulewidth}{0pt}
\renewcommand{\footrulewidth}{0pt}}
\pagestyle{basdepage}

\title{Fiche de Lecture Type}
\author{Xavier Joudiou}
\date{09/04/10}

\begin{document}
  %\thispagestyle{basdepage}
  \fairetitre{Fiche de Lecture Avatar Mobility in U-C NVW }{Xavier Joudiou}{14/04/10}
  \infoFicheLecture{Avatar mobility in user-created networked virtual worlds: measurements, analysis, and implications}{Huiguang Liang, Ransi Nilaksha De Silva, Wei Tsang Ooi, Mehul Motani}{21 mai 2009}{Springer Science + Business Medi, LLC 2009}{Networked virtual environment(NVE), Mobility traces user behavior, Peer-to-Peer,Caching,Pre fetching,Second Life}
	
  \begin{itemize}
  \renewcommand{\labelitemi}{$\Rightarrow$}
	\item beaucoup de partie identique à leur autre papier
	\item user-created networked virtual worlds $\ne$ MMOGs, difference entre FPS et MMOGs, monde très dynamique dans Second Life,pas de visibilité des avatars des autres régions.
	\item sticky cells ?? $\rightarrow$ pause plus longue et predictable $\rightarrow$ super nœud potentiel ??
	\item amélioration de la scalability $\rightarrow$ cluster server, les zones populaires ne varient pas beaucoup au cours du temps $\rightarrow$ utilisation historique, deux états des avatars (lent ou rapide), Second Life prefetch les données dans un cercle autour de l'avatar mais que 18\% du temps efficace, amélioration possible de la gestion du cache de Second Life.
	\item ...
  \end{itemize}

\end{document}  
