\documentclass[11pt,a4paper]{article}

\usepackage{Type_Vocabulaire} %% cibler doc/modules/

\usepackage{fancyhdr}
\fancypagestyle{basdepage}{
\fancyfoot{} % clear all footer fields
\fancyfoot[C]{Stage Lip6 : Amélioration de la réactivité des réseaux p2p pour les MMOGs [\thepage]}
\renewcommand{\headrulewidth}{0pt}
\renewcommand{\footrulewidth}{0pt}}
\pagestyle{basdepage}

\title{Fiche de Vocabulaire Type}
\author{Xavier Joudiou}
\date{09/04/10}

\begin{document}
  %\thispagestyle{basdepage}
  \fairetitre{Fiche de Vocabulaire Voronet}{Xavier Joudiou}{06/04/10}
  \infoFicheVocabulaire{Titre}{Auteur}{Année}
	
  \begin{itemize}
  \renewcommand{\labelitemi}{$\Rightarrow$}
	\item Voronoi tessellations : Diagramme ( ou décomposition ou partition ou polygones )de Voronoï, représente une décomposition particulière de espace métrique déterminée par les distances à un ensemble discret d'objets ( de points ) de l'espace. Il doit son nom au mathématicien russe Georgi Fedoseevich Voronoï (1868 – 1908).
	\item Pastry : Protocole pair à pair utilisant une table de hachage distribuée
	\item Chord : Chord est un projet pair à pair utilisant une topologie en anneau, utilisation d'un table de hachage distribuée.
	\item ...
  \end{itemize}

\end{document}  
