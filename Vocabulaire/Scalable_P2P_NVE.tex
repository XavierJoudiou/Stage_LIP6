\documentclass[11pt,a4paper]{article}

\usepackage{../Ressources/LaTex/Type_Vocabulaire} %% cibler doc/modules/

\usepackage{fancyhdr}
\fancypagestyle{basdepage}{
\fancyfoot{} % clear all footer fields
\fancyfoot[C]{Stage Lip6 : Amélioration de la réactivité des réseaux p2p pour les MMOGs [\thepage]}
\renewcommand{\headrulewidth}{0pt}
\renewcommand{\footrulewidth}{0pt}}
\pagestyle{basdepage}

\title{Fiche de Vocabulaire Type}
\author{Xavier Joudiou}
\date{09/04/10}

\begin{document}
  %\thispagestyle{basdepage}
  \fairetitre{Fiche de Vocabulaire Scalable P2P Networked Virtual Environment}{Xavier Joudiou}{27/04/10}
  \infoFicheVocabulaire{Scalable Peer-to-Peer Networked Virtual Environment}{Shun-Yun Hu, Guna-Ming Liao}{2004}
	
  \begin{itemize}
  \renewcommand{\labelitemi}{$\Rightarrow$}
	\item SIMNET was a wide area network with vehicle simulators and displays for real-time distributed combat simulation: tanks, helicopters and airplanes in a virtual battlefield. SIMNET was developed for and used by the United States military. SIMNET development began in the mid-1980s, was fielded starting in 1987, and was used for training until successor programs came online well into the 1990s.
	\item ...
  \end{itemize}

\end{document}  
