\documentclass{beamer}
%\documentclass[notes]{beamer}
%\documentclass[notes=only]{beamer}
%\documentclass[handout]{beamer}
\usepackage[utf8]{inputenc}
\usetheme{Ilmenau}
\usepackage[frenchb]{babel}    % le documents est en français
\usepackage{amsmath}           % un packages mathématiques
\usepackage{xcolor}            % pour définir plus de couleurs 
\usepackage{graphicx}          % pour insérer des figures
%\usepackage{handoutWithNotes}
%\pgfpagesuselayout{1 on 1 with notes landscape}[a4paper,border shrink=5mm]

		
\usepackage{Slides_08092010}

  \title{Amélioration de la réactivité des réseaux pair à pair pour les MMOGs}
  \author{Xavier Joudiou,\\\tiny{Encadré par: S.Legtchenko \& S.Monnet}}\institute{Université Paris VI, Master SAR}
 % \logo{\includegraphics[scale=0.5]{./Ressources/Images/P2P.png}}
  \date{8 Septembre 2010}

  \begin{document}

  \begin{frame}
  \maketitle
  \end{frame}


  \begin{frame}
  \tableofcontents
  \end{frame}

  \section{Introduction}
  \begin{frame}
  	Présentation des points importants à la compréhension du sujet:\\
	\begin{itemize}
		\item Architecture pair à pair Vs Client-Serveur\\
		\item Définition de l'overlay\\
	\end{itemize}
  \end{frame}

  \begin{frame}
	Architecture pair à pair Vs Client/Serveur\\
	\begin{figure}
	\includegraphics[scale=0.28]{./Ressources/Images/p2p-85145.png}\\
        \label{P2PvsClServ}
        \end{figure}
	\begin{itemize}
		\item Problème du passage à l'échelle de l'architecture Client/Serveur.\\
		\item Solutions p2p existantes pas assez réactives pour assurer une latence suffisante.\\
	\end{itemize}
  \end{frame}
  
  \begin{frame}
	
	\center{Définition de l'overlay}
 	\begin{columns}
          \begin{column}{6,5cm}
	    	\begin{itemize}
		\item Un overlay est un réseau informatique formant un graphe où les liens sont déterminés avec un critère logique.\\
		%\item Réseau physique~$\neq$~Réseau virtuel
		\end{itemize}
	  \end{column}
          \begin{column}{5cm}
        	\begin{figure}
        	  \includegraphics[scale=0.1]{./Ressources/Images/overlay.png}\\
        	  \label{Propa_Algo}
        	\end{figure}
          \end{column}
        \end{columns}
	
  \end{frame}
	
  \section{État de l'art}
  \begin{frame}
	Présentation des  mécanismes importants à la compréhension des solutions proposées:\\
	\begin{itemize}
		\item Solipsis\\
		\item Étude des traces des joueurs de MMOG\\
		\item Blue Banana \\
	\end{itemize}
  \end{frame}

  \subsection{Solipsis}

  \begin{frame}
	%\begin{center}
	%\includegraphics[scale=0.2]{./Ressources/Images/solipsis.png}\\
	%\end{center}
	%\vspace{4mm}
	\begin{columns}
          \begin{column}{6cm}
	Solipsis:\\
	\begin{itemize}
		\item propose un monde virtuel entièrement décentralisé et scalable.\\
		\item met en place un overlay avec une forte malléabilité applicative.\\
		\tiny{
			\begin{itemize}
				\item Un réseau est malléable si sa topologie est dynamiquement déterminé par l'application reposant sur ce réseau.\\
				%\item La topologie s'adapte à l'application, si deux avatars se rapprochent dans le monde virtuel, les nœuds dans le réseau logique doivent devenir progressivement voisins. \\
			\end{itemize}
		}
	\end{itemize}
	\end{column}
        \begin{column}{4cm}
        \begin{figure}
        \includegraphics[scale=0.1]{./Ressources/Images/OverlayMalle_et1.png}\\
        \label{Propa_Algo}
        \end{figure}
        \end{column}
        \end{columns}
  \end{frame}

  \begin{frame}
        %\begin{center}
        %\includegraphics[scale=0.2]{./Ressources/Images/solipsis.png}\\
        %\end{center}
        %\vspace{4mm}
        \begin{columns}
          \begin{column}{6cm}
        Solipsis:\\
        \begin{itemize}
                \item propose un monde virtuel entièrement décentralisé et scalable.\\
                \item met en place un overlay avec une forte malléabilité applicative.\\
                \tiny{
                        \begin{itemize}
                                \item Un réseau est malléable si sa topologie est dynamiquement déterminé par l'application reposant sur ce réseau.\\
                                %\item La topologie s'adapte à l'application, si deux avatars se rapprochent dans le monde virtuel, les nœuds dans le réseau logique doivent devenir progressivement voisins. \\
                        \end{itemize}
                }
        \end{itemize}
        \end{column}
        \begin{column}{4cm}
        \begin{figure}
        \includegraphics[scale=0.1]{./Ressources/Images/OverlayMalle_et2.png}\\
        \label{Propa_Algo}
        \end{figure}
        \end{column}
        \end{columns}
  \end{frame}


  \begin{frame}
	Solipsis introduit deux propriétés:
	 \begin{columns}
          \begin{column}{6cm}
	   \begin{itemize}
                \item \textit{Connaissance locale:}\\\tiny{
                Une entité doit être connectée avec tous ses plus proches voisins, elle peut connaître des entités en dehors de son environnement virtuel local. Toute entité située à l'intérieur de l'environnement d'une entité doit faire parti des voisins de cette entité.}
                \item \normalsize{\textit{Connectivité globale:}}\\\tiny{
                Toute entité virtuelle doit se trouver à l'intérieur de l'enveloppe convexe contenant l'ensemble de ses voisins logiques. \\}
	   \end{itemize}
	\end{column}
        \begin{column}{4cm}
        \begin{figure}
        \includegraphics[scale=0.3]{./Ressources/Images/envelop_convex3.png}\\
        \includegraphics[scale=0.3]{./Ressources/Images/envelop_convex4.png}\\
        \label{Propa_Algo}
        \end{figure}
        \end{column}
        \end{columns}
  \end{frame}

  \subsection{Les traces}
  \begin{frame}
	Des études des traces des joueurs de MMOG, ont permis de faire plusieurs observations sur l'environnement virtuel:
	\begin{itemize}
		\item Présence de zones denses 
		\item Mouvements erratiques dans les zones denses
		\item Mouvements rectilignes et rapides entre les zones denses
	\end{itemize}
	\begin{center}
        \includegraphics[scale=0.35]{./Ressources/Images/trace.png}\\
        \end{center}
  \end{frame}

  \subsection{BlueBanana}
  \begin{frame}
	 Blue Banana introduit trois états, pour un avatar:
        \begin{itemize}
                \item \textbf{H}(alted): l'avatar est immobile.
                \item \textbf{T}(ravelling): l'avatar se déplace rapidement sur la carte et il a une trajectoire droite.
                \item \textbf{E}(xploring): l'avatar est en train d'explorer une zone, sa trajectoire est confuse et sa vitesse est lente.
        \end{itemize}
  \end{frame}

  \begin{frame}
	Mise en place d'un mécanisme d'anticipation des mouvements des avatars.
	\begin{columns}
	  \begin{column}{5cm}
		\begin{itemize}
		  \item Si état \textbf{T}, il cherche des nœuds sur sa trajectoire.
		  \item Evaluation du nœud, propagation de la requête.
		  \item Réponse au nœud qui a demandé le préchargement.
		\end{itemize}
	  \end{column}
	\begin{column}{5cm}
	\begin{figure}
        \includegraphics[scale=0.1]{./Ressources/Images/propagation_algo.png}\\
        \label{Propa_Algo}
        \end{figure}
	\end{column}
	\end{columns}
  \end{frame}

  \section{Les améliorations}
  \begin{frame}
	Durant le stage, plusieurs solutions ont été implémentées:
	\begin{itemize}
		\pause\item Le cache pour les zones denses
		\pause\item Le préchargement amélioré des données
	\end{itemize}
	\pause D'autres solutions ont été étudiées, mais sans être implémentées.
	\begin{itemize}
		\item Mouvements de groupe
		\item Connaissance des routes entre les zones denses
	\end{itemize}
  \end{frame}

  \begin{frame}
	Différentes métriques utilisées pour analyser les résultats:
	\begin{itemize}
		\item Nombre de messages à un instant 
		\item Cohérence de la topologie \\ \textit{\footnotesize{Nombre de nœuds dans la zone de connaissance mais pas dans l'ensemble des voisins}}
	\end{itemize}
	\vspace{5mm}
	En fonction du degré de mobilité.
  \end{frame}

  \section{Le cache pour les zones denses}
  \begin{frame}
	\center{Le cache pour les zones denses}
	\vspace{1cm}
	\begin{itemize}
		\item Explications du cache pour les zones denses
		\item Les résultats 
		\item Conclusion sur le cache
	\end{itemize}
  \end{frame}
  
  \subsection{Explications du cache pour les zones denses}
  \begin{frame}
	Comment fonctionne le cache?
	\begin{itemize}
 		\item Chaque nœud de l'environnement a un cache.
		\item Il est utilisé seulement par les nœuds en état \textbf{E}(xploring).
		\item Deux types de cache mis en place (retour simple et retour multiple).
	\end{itemize}
  \end{frame}

  \begin{frame}
	Trois types de recherche dans le cache:
	\vspace{3mm}
	\tiny{
	\begin{table}
  		\begin{center}
    		 \begin{tabular}{|c|c|c|c|}
      		 \hline
      		 N & Critère de sélection & Avantages & Inconvénients\\
      		 \hline
        	 1 & Comparaison distances & Simplicité & Distance~$\ne$~utile, aide pas enveloppe\\
        	 2 & Aide enveloppe & + Enveloppe OK & - bon règles Solipsis\\
        	 3 & Zone de connaissance & Simplicité & aide pas enveloppe\\
      		 \hline
    		 \end{tabular}
  		\end{center}
	\end{table}
	}
	\begin{itemize}
		\item La version 3 a été conservé pour les tests finaux.

	\end{itemize}
	\center{\includegraphics[scale=0.13]{./Ressources/Images/cacheReconstructEnvelop2.png}\\}

  \end{frame}
 
  \begin{frame}
  	\only{\begin{figure}
        \includegraphics[scale=0.08]{./Ressources/Images/etape1.png}\\
        \label{etape1}
        \end{figure}}
  \end{frame}
	
  \begin{frame}	
  	\only{\begin{figure}
        \includegraphics[scale=0.08]{./Ressources/Images/etape2.png}\\
        \label{etape2}
        \end{figure}}
  \end{frame}

  \begin{frame}
  	\only{\begin{figure}
        \includegraphics[scale=0.08]{./Ressources/Images/etape3.png}\\
        \label{etape3}
        \end{figure}}


  \end{frame}

  \begin{frame}
	Différents mécanismes pour le cache:
	\begin{itemize}
		\item Mise à jour des données du cache
		\item Contact un nœud du cache s'il est là depuis longtemps
		\item Aide les nœuds voisins lors de recherche de nœud	
	\end{itemize}
	\footnotesize{
	\begin{table}[!h]
  	\begin{center}
    	\begin{tabular}{|c|c|}
        \hline
      	 Paramètre & Valeur\\
      	\hline
     	 Taille du cache & 25\\
     	 Limite de distance &  1500\\
     	 Limite de temps & 1500\\
     	 Contact Nœud & Faux\\
     	 Mise à jour du cache & Faux\\
     	 Aide aux voisins & Vrai\\
      	\hline
    	\end{tabular}
  	\end{center}
	\end{table}
	}
  \end{frame}

 
  \subsection{Résultats pour le cache}
  \begin{frame}
	\begin{center}
	Nombre de messages 
	\end{center}
	\begin{columns}
         \begin{column}{5cm}
          \includegraphics[scale=0.25]{./Ressources/Images/Courbes_Final_Rapport/Nombre_Messages_Caches.png}\\
         \end{column}
         \begin{column}{5cm}
	\footnotesize{ \begin{table}[!h]
                \begin{center}
                \begin{tabular}{|c|c|}
                \hline
                Solution & Nombre de messages \\
                \hline
                Cache simple &  5\% de msg en moins\\
                Cache multiple &  5\% de msg en moins\\
                \hline
                \end{tabular}
                \end{center}
        \end{table}}
         \end{column}
        \end{columns}
        \begin{itemize}\footnotesize{
                \item Moins de messages le cache s'utilise immédiatement sans message.
        }
        \end{itemize}
  \end{frame}

  \begin{frame}
        \begin{center}
        Cohérence de la topologie
        \end{center}
        \begin{columns}
         \begin{column}{5cm}
          \includegraphics[scale=0.25]{./Ressources/Images/Courbes_Final_Rapport/Topology_Coherence_Caches.png}\\
         \end{column}
         \begin{column}{5cm}
	\footnotesize{
         \begin{table}[!h]
                \begin{center}
                \begin{tabular}{|c|c|}
                \hline
                Solution & Cohérence topologie \\
                \hline
                Cache simple & Équivalente \\
                Cache multiple & 3\% de gains \\
                \hline
                \end{tabular}
                \end{center}
        \end{table}}
         \end{column}
        \end{columns}
        \begin{itemize}\footnotesize{
                \item Gain sur la cohérence de la topologie si retour multiple.
        }
        \end{itemize}
  \end{frame}
	
  \subsection{Conclusion sur le cache des zones denses}
  \begin{frame}
  	La mise en place du cache permet:\\
	\begin{itemize}
		\item d'économiser des messages.\\
		\item d'améliorer la cohérence de la topologie.\\
	\end{itemize}
	\vspace{5mm}
	Amélioration possible en testant toutes les combinaisons de paramètres (mise à jour, contact d'un nœud, taille du cache, etc).\\
  \end{frame}



  \section{L'amélioration du préchargement des données}
  \begin{frame}
	\center{L'amélioration du préchargement des données}
	\vspace{1cm}
	\begin{itemize}
		\item Explications sur le préchargement amélioré
		\item Les résultats 
		\item Conclusion sur le préchargement
	\end{itemize}
  \end{frame}

  \subsection{Explications sur l'amélioration du préchargement des données}
  \begin{frame}
	\begin{columns}
        \begin{column}{7cm}
	\textbf{Situation:} Le préchargement de Blue Banana prend tous les nœuds, à bonne distance, dans le cône.\\
	\vspace{5mm}
	\textbf{Problème:} Des nœuds inutiles sont préchargés.\\
	\vspace{5mm}
	\textbf{Solution:} Choisir plus finement les nœuds qui vont être sélectionnés.\\
	\vspace{5mm}
	\textbf{Comment:} Regarder la direction des nœuds et leur vitesse.\\
         \end{column}
         \begin{column}{3cm}
          \includegraphics[scale=0.1]{./Ressources/Images/prefetchNormal.png}\\
         \end{column}
        \end{columns}
  \end{frame}

  \begin{frame}
	\begin{columns}
         \begin{column}{5cm}
	 Préchargement si:
          \begin{itemize}
		\item l'angle du nœud est proche du nœud courant
		\item \textit{Somme des normes des vecteurs~$\ge$~Norme du vecteur de prefetch +/- $\Delta$}
		\item l'angle du nœud est inverse par mais sa norme est inférieure à celle du nœud courant
	  \end{itemize}
	 \end{column}
         \begin{column}{6cm}
          \includegraphics[scale=0.11]{./Ressources/Images/prefetchaV1.png}\\
         \end{column}
        \end{columns}

  \end{frame}
	
  \subsection{Résultats pour l'amélioration du préchargement des données}
  \begin{frame}
	\begin{center}
	Nombre de messages 
	\end{center}
	\begin{columns}
         \begin{column}{5cm}
          \includegraphics[scale=0.25]{./Ressources/Images/Courbes_Final_Rapport/Nombre_Messages_Prefetchs.png}\\
         \end{column}
         \begin{column}{5cm}
		\begin{table}[!h]
                \begin{center}
                \begin{tabular}{|c|c|}
                \hline
                Solution & Nombre de messages \\
                \hline
                Normal &  8\% de msg en plus\\
                Amélioré &  4\% de msg en plus\\
                \hline
                \end{tabular}
                \end{center}
        \end{table}
         \end{column}
        \end{columns}
	\begin{itemize}\footnotesize{
		\item Gain en terme de messages car préchargement plus efficace, et donc moins de recherche de voisins.
		}
	\end{itemize}
  \end{frame}
	
  \begin{frame}
        \begin{center}
        Cohérence de la topologie
        \end{center}
        \begin{columns}
         \begin{column}{5cm}
          \includegraphics[scale=0.25]{./Ressources/Images/Courbes_Final_Rapport/Topology_Coherence_Prefetchs.png}\\
         \end{column}
         \begin{column}{5cm}
		\begin{table}[!h]
                \begin{center}
                \begin{tabular}{|c|c|}
                \hline
                Solution & Cohérence topologie \\
                \hline
                Normal & 15\% de gains \\
                Amélioré & 16\% de gains \\
                \hline
                \end{tabular}
                \end{center}
        \end{table}
         \end{column}
        \end{columns}
	\begin{itemize}\footnotesize{
		\item Léger gain sur la cohérence de la topologie car élimination du préchargement de certains nœuds inutiles.
		}
	\end{itemize}
  \end{frame}

  \subsection{Conclusion sur l'amélioration du préchargement des données}
  \begin{frame}
	Notre amélioration du préchargement permet:
	\begin{itemize}
		\item d'économiser des messages par rapport au préchargement normal.\\
		\item d'améliorer légèrement la cohérence de la topologie.\\
	\end{itemize}
	\vspace{5mm}
	Possibles améliorations du préchargement en regardant d'autres paramètres, comme la distance avec les nœuds.
  \end{frame}
	
  
  \section{Conclusion}
  \begin{frame}
	\begin{itemize}
	\item Les solutions ont permis d'améliorer la réactivité des réseaux pair à pair pour les MMOGs.\\
	\item Meilleur cohérence de la topologie et moins de message que dans Blue Banana.
	\item Perspectives :\\
	\begin{itemize}
	\footnotesize{
	\item Meilleure utilisation des mécanismes du cache
	\item Amélioration du préchagememnt
	\item Mouvements de groupe\\
	\item Route entre les zones denses.\\}
	\end{itemize}
	\end{itemize}
  \end{frame}
	

  \begin{frame}
	\begin{center}
	Merci.\\
	\vspace{1cm}	
	Questions?
	\end{center}
  \end{frame}  

  \end{document}
