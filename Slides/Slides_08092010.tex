\documentclass{beamer}
\usepackage[utf8]{inputenc}
\usetheme{Ilmenau}
\usepackage[frenchb]{babel}    % le documents est en français
\usepackage{amsmath}           % un packages mathématiques
\usepackage{xcolor}            % pour définir plus de couleurs 
\usepackage{graphicx}          % pour insérer des figures
		
\usepackage{Slides_08092010}

  \title{Amélioration de la réactivité des réseaux pair à pair pour les MMOGs}
  \author{Xavier Joudiou,\\\tiny{Encadré par: S.Legtchenko \& S.Monnet}}\institute{Université Paris VI, Master SAR}
 % \logo{\includegraphics[scale=0.5]{./Ressources/Images/P2P.png}}
  \date{8 Septembre 2010}

  \begin{document}

  \begin{frame}
  \maketitle
  \end{frame}


  \begin{frame}
  \tableofcontents
  \end{frame}

  \section{Introduction}
  \begin{frame}
	Nous présenter les concepts nécessaires à une bonne compréhension des solutions qu'il nous a été possible d'implémenter.
  \end{frame}
	
  \section{État de l'art}
  \begin{frame}
	Présentation des différents solutions sur lesquels, nos solutions s'appuient.
	\begin{itemize}
		\item Solipsis
		\item Blue Banana
	\end{itemize}
  \end{frame}

  \subsection{Solipsis}
  \begin{frame}
	Explication de Solipsis
  \end{frame}

  \begin{frame}
	Explication de Solipsis Suite
  \end{frame}

  \subsection{BlueBanana}
  \begin{frame}
	Blue Banana c cool
  \end{frame}

  \begin{frame}
	Blue Banana c trop cool
  \end{frame}

  \section{Les améliorations}
  \begin{frame}
	Durant le stage, plusieurs solutions ont été implémentées:
	\begin{itemize}
		\item Le cache pour les zones dense
		\item Le préchargement amélioré des données
	\end{itemize}
  \end{frame}

  \subsection{Le cache pour les zones denses}
  \begin{frame}
  \end{frame}

  \begin{frame}
  \end{frame}
	
  \subsection{Le préchargement amélioré des données}
  \begin{frame}
  \end{frame}
	
  \begin{frame}
  \end{frame}
	
  \section{Résultats des solutions implémentées}
  \subsection{Métriques pour les tests}
  \begin{frame}
  \end{frame}

  \subsection{Résultats pour le cache}
  \begin{frame}
  \end{frame}
	
  \subsection{Résultats pour le préchargement amélioré}
  \begin{frame}
  \end{frame}
  
  \section{Conclusion}
  \begin{frame}
  \end{frame}
  
  \end{document}
